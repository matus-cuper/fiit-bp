% Bakalársky projekt 2016/2017
% Matúš Cuper

%-------------------------------------------------------------------------------
%   PACKAGES AND DOCUMENT CONFIGURATION
%-------------------------------------------------------------------------------

\documentclass[a4paper,slovak,12pt,appendix]{article}

% \usepackage{float}
\usepackage[slovak]{babel}                                                      % title in Slovak
\usepackage[utf8]{inputenc}                                                     % supported special Slovak characters
\usepackage[T1]{fontenc}                                                        % supported word wrapping on the end of line on speciacl Slovak characters
% \usepackage{url}
\usepackage{times}																															% Times New Roman
\usepackage[unicode]{hyperref}																									% enable hyper references in table of content
%\usepackage{indentfirst}                                                        % indent first line after section, but carefully, indent everything and everywhere
\usepackage{amsmath}                                                            % for fractions
\usepackage{amssymb}                                                            % for real numbers sign
\usepackage{algorithm, algpseudocode}                                           % for pseudocode writting
\usepackage{graphicx}                                                           % enable figures
\graphicspath{ {images/} }                                                      % set path for figures
\hypersetup{																																		% with default colours for links
    colorlinks,
		pageanchor=false,
    citecolor=black,
    filecolor=black,
    linkcolor=black,
    urlcolor=black
}

% \usepackage{cite}
% \usepackage{times}
% \usepackage[dvips,dvipdfm,a4paper,centering,textwidth=14cm,top=4.6cm,headsep=.6cm,footnotesep=1cm,footskip=0.6cm,bottom=3.8cm]{geometry}
\usepackage[a4paper, centering,
											left=35mm, top=25mm, right=20mm, bottom=25mm]{geometry}		% set page margins

%-------------------------------------------------------------------------------
%   TITLE PAGES
%-------------------------------------------------------------------------------

\begin{document}
\begin{titlepage}
	\centering
	{\Large Slovenská technická univerzita v Bratislave \par}
	{\Large Fakulta informatiky a informačných technológií \par}
  \vspace{0.5cm}
  {\normalsize Evidenčné číslo: FIIT-000-00000 \par}
	\vspace{7cm}
  {\large Matúš Cuper \par}
  \vspace{0.5cm}
	{\LARGE Optimalizácia konfiguračných parametrov predikčných metód \par}
	\vspace{0.5cm}
	{\large Bakalárska práca \par}
	\vspace{7cm}
  \flushleft
	{\large Vedúci práce: Ing. Marek Lóderer \par}
  \vspace{0.5cm}
  {\large máj 2017 \par}
	\vfill
\end{titlepage}

\begin{titlepage}
	\centering
  {\Large Slovenská technická univerzita v Bratislave \par}
	{\Large Fakulta informatiky a informačných technológií \par}
  \vspace{0.5cm}
  {\normalsize Evidenčné číslo: FIIT-000-00000 \par}
	\vspace{7cm}
  {\large Matúš Cuper \par}
  \vspace{0.5cm}
	{\LARGE Optimalizácia konfiguračných parametrov predikčných metód \par}
	\vspace{0.5cm}
	{\large Bakalárska práca \\}
	\vspace{7cm}
  \flushleft
  {\normalsize Študijný program: Informatika \par}
	{\normalsize Študijný odbor: 9.2.1 Informatika \par}
	{\normalsize Miesto vypracovania: Ústav informatiky a softvérového inžinierstva, FIIT STU Bratislave \par}
	{\normalsize Vedúci práce: Ing. Marek Lóderer \par}
  \vspace{0.5cm}
  {\normalsize máj 2017 \par}
\end{titlepage}

%-------------------------------------------------------------------------------
%   ANOTATION
%-------------------------------------------------------------------------------

% \newpage\null\thispagestyle{empty}\newpage

\begin{titlepage}
\begin{center}
  {\small Slovenská technická univerzita v Bratislave \par}
  {\small \textbf{FAKULTA INFORMATIKY A INFORMAČNÝCH TECHNOLÓGIÍ}}
  \rule{\textwidth}{1pt}

  \vspace*{1.5cm}
  \begin{Large}
    \textbf{Anotácia} \par
  \end{Large}
\end{center}
{Slovenská technická univerzita v Bratislave \par}
{FAKULTA INFORMATIKY A INFORMAČNÝCH TECHNOLÓGIÍ \par}
{Študijný program: Informatika \par}
{Autor: Matúš Cuper \par}
{Bakalárska práca: Optimalizácia konfiguračných parametrov predikčných metód \par}
{Vedúci práce: Ing. Marek Lóderer \par}
{máj 2017 \\} \\
V práci sme sa zamerali na problémy vznikajúce pri predikcii časových radov.
V súčasnosti existuje veľké množstvo metód, ktoré nám zabezpečujú predpoveď
sledovanej veličiny s prijateľne malou odchýlkou na krátke obdobie v blízkej
budúcnosti. Cieľom bakalárskej práce bolo vytvoriť systém, ktorý používateľovi
poskytne jednoduché rozhranie pre porovnanie jednotlivých predikčných
algoritmov nad množinou dát, ktorú si sám zvolí. Hľadanie ich optimálneho
nastavenia sa vykonáva pomocou optimalizačných algoritmov založených na
správaní sa živočíchov v prírode.

V práci sme analyzovali a opísali množinu predikčných a optimalizačných
algoritmov. Navrhli sme systém na hľadanie optimálnych parametrov predikčných
metód, čím sme výrazne ovplyvnili ich presnosť. Systém bol implementovaný
v programovacom jazyku R a na vytvorenie používateľského rozhrania bola použitá
knižnica Shiny. Optimalizácie sme vykonávali nad dátovými množinami v doméne
energetiky. Výsledný systém umožňuje používateľovi využívať silu predikčných
algoritmov a nájsť ich optimálne parametre pre zabezpečenie čo najpresnejšej
predikcie.
\end{titlepage}

\begin{titlepage}
\begin{center}
  {\small Slovak University of Technology Bratislava \par}
  {\small \textbf{FACULTY OF INFORMATICS AND INFORMATION TECHNOLOGIES}}
  \rule{\textwidth}{1pt}

  \vspace*{1.5cm}
  \begin{Large}
    \textbf{Annotation} \par
  \end{Large}
\end{center}
{Slovak University of Technology Bratislava \par}
{FACULTY OF INFORMATICS AND INFORMATION TECHNOLOGIES \par}
{Degree Course: Computer Science \par}
{Author: Matúš Cuper \par}
{Bachelor thesis: Optimizing configuration parameters of prediction methods \par}
{Supervisor: Ing. Marek Lóderer \par}
{May 2017 \\} \\
In the thesis we focused on problems, which appear in time series prediction.
In present there are many methods, which predict observed value with acceptable
small deviation for short time period in near future. The aim of bachelor
thesis was creating system, which provides simple user interface to compare
chosen prediction algorithms on dataset, which is chosen by user. Looking for
their optimal setup is made by optimization algorithm based on nature-inspired
behavior.

In the thesis we analyzed and described set of prediction and optimization
algorithms. We designed system for searching optimal parameters of prediction
methods, which influence their accuracy significantly. System was implemented
in programming language R and for creating user interface was used library
Shiny. Optimization was provided on dataset in energetics domain. The final
system provides to user to use force of prediction algorithms and find out
their optimal parameters for the most accurate prediction.
\end{titlepage}

%-------------------------------------------------------------------------------
%   Declaration
%-------------------------------------------------------------------------------

\begin{titlepage}
\vspace*{15cm}
\begin{large}
  \noindent \textbf{ČESTNÉ PREHLÁSENIE} \par
\end{large}
\vspace*{0.5cm}
\noindent
Čestne prehlasujem, že bakalársku prácu som vypracoval samostatne pod vedením
vedúceho bakalárskej práce a s použitím odbornej literatúry, ktorá je uvedená
v zozname použitej literatúry. \\
\vspace*{0.5cm}\\
\hspace*{10cm}............................\\
\hspace*{10.7cm} Matúš Cuper
\end{titlepage}

\begin{titlepage}
\vspace*{15cm}
\begin{large}
  \noindent \textbf{POĎAKOVANIE} \par
\end{large}
\vspace*{0.5cm}
\noindent
Ďakujem vedúcemu bakalárskej práce Ing. Marekovi Lódererovi za odborné vedenie,
cenné rady a pripomienky pri spracovaní bakalárskej práce.
\end{titlepage}

%-------------------------------------------------------------------------------
%   Table of contents
%-------------------------------------------------------------------------------

\newpage
\tableofcontents
\thispagestyle{empty}                                                           % removes page numbering from table of contents page

\newpage
\listoffigures
\thispagestyle{empty}

\newpage
\listoftables
\thispagestyle{empty}

%-------------------------------------------------------------------------------
%   Chapter 1 - Introduction
%-------------------------------------------------------------------------------

\newpage
\setcounter{page}{1}
\section{Úvod}
V súčasnosti sme obklopení množstvom zariadení, ktoré merajú a zhromažďujú
informácie z iných zariadení. Príkladom môžu byť rôzne mobilné aplikácie,
meteorologické stanice alebo inteligentné merače merajúce spotrebu elektrickej
energie, vody či plynu. Namerané dáta sa môžu meniť v závislosti od napr.
hodiny, dňa alebo počasia. Samozrejme existujú aj merania, ktoré sú ovplyvnené
ľudským správaním, ktoré sa môže líšiť v závislosti od vyššie uvedených
faktorov, ale aj faktorov ako sú kultúrne tradície, zvyky či náboženstvo. Na
základe nameraných dát vieme vytvoriť predpoveď, ktorá opäť môže ovplyvniť
ostatné faktory a celé predpovedanie sa tak stáva opäť komplexnejším. Ak sú
dáta merané v pravidelných intervaloch a konzistentne, nazývame ich časový rad.

V práci sme sa zamerali na predpovedanie veličín na základe ich historických
meraní. Ostatné faktory pri tom neboli brané do úvahy. Tým sa stáva
predpovedanie jednoduchšie a menej presné, čím vzniká priestor pre hlavný zámer
práce, optimalizovanie predikčných metód na základe ich vstupných parametrov.
Získame tým presnejšiu predpoveď ako keby sme nastavenie predikčných algoritmov
nechali na náhode alebo vlastnom úsudku.

Optimalizačné algoritmy, tak ako ich názov napovedá, slúžia na nájdenie
optimálneho riešenia. Nájdenie najlepšieho riešenia požaduje preskúmanie
všetkých možností. Stáva sa tak pomalým a výpočtovo náročným. Optimalizačné
algoritmy rýchlo nájdu riešenie, ktoré je pre potreby našej práce postačujúce.
Optimalizačné algoritmy môžeme rozdeliť do viacerých skupín, v práci sa však
zameriavame prednostne na prírodne inšpirované algoritmy, ktoré sú jednou
z najefektívnejších podskupín.

Výsledný systém poskytuje používateľovi webové grafické rozhranie, pomocou
ktorého môže predpovedať hodnoty vložených časových radov. Má možnosť zvoliť si
medzi viacerými predikčnými a optimalizačnými algoritmami. Rozhranie poskytuje
používateľovi základné informácie o algoritmoch, ako aj vysvetlenie efektov
jednotlivých vstupných parametrov metód. Je na zvážení používateľa, aké
parametre zvolí pre optimalizačné algoritmy. Vstupné parametre predikčných
metód budú zvolené optimálne na základe výstupných hodnôt optimalizačných
algoritmov. Používateľ má k dispozícií vyhodnotenie predpovede, ktoré porovnáva
predpovedanú a skutočnú hodnotu rôznymi metrikami.

Celá práca je rozdelená do niekoľkých kapitol. V
kapitole~\ref{problem-analysis} sme sa zamerali na analýzu problému. Definovali
sme kľúčové pojmy, použité metódy, opísali sme časové rady, ich vlastnosti,
rozdelili sme predikčné a optimalizačné algoritmy do skupín.
Kapitola~3 popisuje funkcionalitu, ktorú systém poskytuje                       % \ref{specification}
používateľom. V kapitole~4 sa zameriavame na návrh systému                      % \ref{solution-design}
a v kapitole~5 už samotnou implementáciou systému v jazyku R.                   % \ref{implementation}
Výsledky práce sú zhodnotené v kapitole~6.                                      % \ref{evaluation}

%-------------------------------------------------------------------------------
%   Chapter 2 - Problem analysis
%-------------------------------------------------------------------------------

\newpage
\section{Analýza problému}
\label{problem-analysis}
Predpovedanie spotreby elektrickej energie je kľúčovou činnosťou pre plánovanie
a prevádzkovanie rôznych elektronických zariadení. Hľadaný vzor pre časové
rady spotreby elektrickej energie je často komplexný a je zložité ho nájsť aj
kvôli faktorom ako sú napr. zmeny cien elektrickej energie na trhu. Preto sa
stáva implementácia vhodného modelu zaručujúceho presnú predpoveď
náročnou~\cite{Mahalakshmi2016}.

Práve preto vznikajú rôzne metódy na predpovedanie časových radov, ktoré sú
bližšie popísané v nasledujúcich podkapitolách. Väčšina z nich poskytuje
niekoľko vstupných parametrov ako rozhranie pre vnútorné nastavenie algoritmov.
Vďaka tomu máme možnosť ovplyvňovať mieru natrénovania modelu, veľkosť chyby
predikcie alebo dĺžku obdobia, na ktorom budeme model trénovať. Nastavenie
týchto parametrov sa môže pri rôznych datasetoch líšiť, a preto neexistuje
univerzálne riešenie. Hľadať riešenie pomocou prírodne inšpirovaných
algoritmov je efektívne a nájdené riešenie je optimálne. Ďalej sú v tejto
kapitole opísané spôsoby merania chýb, ktoré slúžia na vyhodnotenie efektivity
a správnosti nájdeného riešenia.

%-------------------------------------------------------------------------------
%   Time series
%-------------------------------------------------------------------------------

\subsection{Časové rady}
Časový rad je množina dátových bodov nameraná v čase postupne za sebou.
Matematicky je definovaný ako množina vektorov $x(t)$, kde $t$ reprezentuje
uplynulý čas. Premenná $x(t)$ je považovaná za náhodnú premennú.
Merania v časových radoch sú usporiadané v chronologickom
poradí~\cite{Agrawal2013}.

Časové rady delíme na spojité a diskrétne. Pozorovania pri spojitých časových
radoch sú merané v každej jednotke času, zatiaľ čo diskrétne obsahujú iba
pozorovania v diskrétnych časových bodoch. Hodnoty toku rieky, teploty
či koncentrácie látok pri chemickom procese môžu byť zaznamenané ako spojitý
časový rad. Naopak, populácia mesta, produkcia spoločnosti alebo kurzy mien
reprezentujú diskrétny časový rad. Vtedy sú pozorovania oddelené rovnakými
časovými intervalmi, napr. rokom, mesiacom či dňom~\cite{Agrawal2013}. V našom
prípade sú namerané dáta dostupné každú celú štvrťhodinu.

\subsubsection{Analýza časových radov}
\label{time-series-analysis}
V praxi je vhodný model napasovaný do daného časového radu a zodpovedajúce
parametre sú predpovedané na základe známych dát. Pri predpovedaní časových
radov sú dáta z predchádzajúcich meraní zhromažďované a analyzované za účelom
navrhnutia vhodného matematického modelu, ktorý zaznamenáva proces generovania
dát pre časové rady. Pomocou tohto modelu sú predpovedané hodnoty budúcich
meraní. Takýto prístup je užitočný, keď nemáme veľa poznatkov o vzore
v meraniach idúcich za sebou alebo máme model, ktorý poskytuje nedostatočne
uspokojivé výsledky~\cite{Agrawal2013}.

Cieľom predikcií časových radov je predpovedať hodnotu premennej v budúcnosti
na základe doteraz nameraných dátových vzoriek. Matematicky zapísané ako
\begin{equation}
  \hat{x}(t+\Delta_t) = f(x(t-a), x(t-b), x(t-c), ...)
  \label{eq-series}
\end{equation}
Hodnota $\hat{x}$ vo vzorci~\ref{eq-series} je predpovedaná hodnota
jednorozmerného diskrétneho časového radu $x$. Úlohou je nájsť takú funkciu
$f(x)$, pre ktorú bude $\hat{x}$ predstavovať predpovedanú hodnotu časového
radu. Táto funkcia by mala predpovedať hodnoty v budúcnosti konzistentne
a objektívne~\cite{Sapankevych2009}.

Časové rady sú najčastejšie vizualizované ako graf, kde pozorovania sú na
osy $y$ a plynúci čas na osy $x$. Pre lepšie vysvetlenie časových radov, budú
nasledujúce odstavce obsahovať aj obrázky zobrazujúce vygenerovanú dátovú
množinu.

\subsubsection{Zložky časových radov}
Pri predpovedaní časových radov ako napr. meraní odberu elektrickej energie
vznikajú 2 typy trendov. Prvým typom je trvalá alebo dočasná zmena spôsobená
ekonomickými alebo ekologickými faktormi. Druhým typom je sezónna zmena,
spôsobená zmenami ročných období a množstvom denného svetla. Môžeme ju pozorovať
na úrovni dní, týždňov alebo rokov. Veličina, ktorú sa snažíme predpovedať
postupne mení svoje správanie a model sa tak stáva nepresným. Kvôli tomu je
nutné v každom modely rozdeľovať tieto typy tendencií, aby sme vedeli model
zmenám prispôsobiť~\cite{Grmanova2016}.

Vo všeobecnosti sú časové rady zložené zo 4 hlavných zložiek, ktoré môžeme
odlíšiť od pozorovaných dát. Jedná sa o trendovú, cyklickú, sezónnu
a reziduálnu zložku~\cite{Agrawal2013}.

\paragraph{Trendová zložka} predstavuje správanie časového radu v dlhodobom
časovom horizonte. Z tohto pohľadu má časový rad tendenciu klesať, rásť alebo
stagnovať. Príkladom môže byť nárast populácie či klesajúca
úmrtnosť~\cite{Agrawal2013}.

\begin{figure}[!ht]
  \centering
  \includegraphics[width=0.65\textwidth]{trend_component.pdf}
  \caption{Príklad trendovej zložky časového radu.}
  \label{fig-trend-comp}
\end{figure}

\paragraph{Cyklická zložka} je spôsobená zmenami, ktoré sa cyklicky opakujú.
Dĺžka periódy je 2 a viac rokov, čo zodpovedá strednodobému časovému horizontu.
Táto zložka je zastúpená najmä pri ekonomických časových radoch napríklad
podnikateľský cyklus pozostávajúci zo 4 fáz, ktoré sa stále
opakujú~\cite{Agrawal2013}.

\begin{figure}[H]
  \centering
  \includegraphics[width=0.7\textwidth]{season_component.pdf}
  \caption{Príklad sezónnej zložky časového radu.}
  \label{fig-season-comp}
\end{figure}

\begin{figure}[H]
  \centering
  \includegraphics[width=0.7\textwidth]{random_component.pdf}
  \caption{Príklad reziduálnej zložky časového radu.}
  \label{fig-random-comp}
\end{figure}

\paragraph{Sezónna zložka} predstavuje kolísanie časových radov počas ročných
období. Dôležitými faktormi pri tom sú napr. klimatické podmienky, tradície
alebo počasie. Napríklad predaj zmrzliny sa v lete zvyšuje, ale počet
predaných lyžiarskych súprav klesá~\cite{Agrawal2013}.

\paragraph{Reziduálna zložka} predstavuje veličinu, ktorá nemá žiadny
opakovateľný vzor a ani dlhodobý trend. V časových radoch má nepredvídateľný
vplyv na pozorovanú veličinu. V štatistike zatiaľ nie je definovaná metóda jej
merania. Označuje sa aj ako náhodná zložka alebo biely šum. Je spôsobená
nepredvídateľnými a nepravidelnými udalosťami~\cite{Agrawal2013}.

Vo všeobecnosti sa pre tieto 4 zložky používajú 2 rôzne modely. Je to
multiplikatívny model a aditívny model.
\begin{equation}
  \begin{split}
    Y(t) = T(t) \times S(t) \times C(t) \times I(t)
    \\
    Y(t) = T(t) + S(t) + C(t) + I(t)
  \end{split}
  \label{eq-ts-models}
\end{equation}
Vo vzorci~\ref{eq-ts-models} predstavuje $Y(t)$ meranie v čase $t$. Premenné
$T(t)$, $S(t)$, $C(t)$ a $I(t)$ sú zložkami trendu, sezónnosti,
cyklu a náhodnosti. Multiplikatívny model, zobrazený na
obrázku~\ref{fig-multi-model} je založený na predpoklade, že časové rady môžu
byť na sebe závislé a môžu byť ovplyvňované medzi sebou, zatiaľ čo aditívny
model, zobrazený na obrázku~\ref{fig-add-model} predpokladá nezávislosť
zložiek~\cite{Agrawal2013}.

\begin{figure}[!ht]
  \centering
  \includegraphics[width=0.8\textwidth]{multi_model.pdf}
  \caption{Príklad multiplikatívneho modelu}
  \label{fig-multi-model}
\end{figure}

\begin{figure}[!ht]
  \centering
  \includegraphics[width=0.8\textwidth]{add_model.pdf}
  \caption{Príklad aditívneho modelu.}
  \label{fig-add-model}
\end{figure}

%-------------------------------------------------------------------------------
%   Analysis of prediction algorithms
%-------------------------------------------------------------------------------

\subsection{Analýza predikčných algoritmov}
Na základe množstva výskumov, ktoré analyzovali predikčné algoritmy, sme si
zvolili reprezentatívnu vzorku algoritmov. Našou snahou bude rôzne predikčné
metódy optimalizovať pomocou optimalizačných algoritmov. Pred tým je potrebné
analyzovať a pochopiť použité predikčné metódy. Taktiež je potrebné
identifikovať ich vstupné parametre, keďže ich neskôr budeme optimalizovať.

%-------------------------------------------------------------------------------
%   Linear regression
%-------------------------------------------------------------------------------

\subsubsection{Lineárna regresia}
Najpoužívanejšia štatistická metóda, ktorá modeluje vzťah závislej premennej
a vysvetľujúcej premennej. Závislú premennú predstavuje veličina, ktorú sa
snažíme predpovedať, čo je v našom prípade spotreba elektrickej energie.
Vysvetľujúca premenná v sebe zahŕňa rôzne faktory, ktoré ovplyvňujú závislú
premennú. Môžeme si pod tým predstaviť deň v týždni, počasie, tradície alebo rôzne
udalosti, ktoré majú vplyv na predpoveď.~\cite{KumarSingh2013, Mahalakshmi2016}.

Predpokladajme typický regresný problém. Dáta pozostávajúce z množiny \textit{n}
meraní majú formát $\left\{(x_1, f(x_1)), ..., (x_n, f(x_n))\right\}$.
Úlohou regresie je odvodiť funkciu $\hat{f}$ z dát, kde
\begin{equation}
  \hat{f} : X \to \mathbb{R} \text{, kde, } \hat{f}(x) = f(x), \forall x \in X,
  \label{eq-regresia}
\end{equation}
Funkcia $f$ vo vzorci~\ref{eq-regresia} reprezentuje reálnu neznámu
funkciu. Algoritmus použitý na odvodenie funkcie $\hat{f}$ sa nazýva
indukčný algoritmus. Funkcia $\hat{f}$ sa nazýva model alebo
prediktor. Obvykle je úlohou regresie minimalizovať odchýlku funkcie pre
štvorcovú chybu, konkrétne strednú štvorcovú chybu MSE~\cite{Mendes-Moreira2012}.

Keďže časový rad pozostáva z viacerých zložiek, môžeme ho zapísať ako funkciu
$L(t)$ definovanú ako
\begin{equation}
  L(t) = Ln(t) + \sum a_i x_i(t) + e(t)
  \label{eq-odber}
\end{equation}
Vo vzorci~\ref{eq-odber} funkcia $Ln(t)$ predstavuje odber elektrickej energie
v čase $t$. Hodnota $a_i$ je odhadovaný pomaly meniaci sa koeficient. Faktory
$x_i(t)$ nezávisle vplývajú na spotrebu elektrickej energie. Môže sa jednať
napr. o počasie alebo zvyky ľudí. Komponent $e(t)$ je biely šum, ktorý má nulovú
strednú hodnotu a pevnú varianciu. Číslo $n$ je počet meraní, obvykle 24
alebo 168, v našom prípade 96 meraní počas jedného dňa~\cite{KumarSingh2013}.

\paragraph{Lineárna regresná analýza} je štatistická metóda používaná na
modelovanie vzťahov, ktoré môžu existovať medzi veličinami. Nachádza súvislosti
medzi závislou premennou a potenciálnymi vysvetľujúcimi premennými. Používame
pri tom vysvetľujúce premenné, ktoré môžu byť namerané súčasne so závislými
premennými alebo aj premenné z úplne iných zdrojov. Regresná analýza môže byť
tiež použitá na zlúčenie trendu a sezónnych zložiek do modelu. Keď je raz model
vytvorený, môže byť použitý na zásah do spomínaných vzťahov alebo, v prípade
dostupnosti vysvetľujúcich premenných, na vytvorenie predikcie~\cite{Liu1992}.

\paragraph{Viacnásobná lineárna regresia} sa snaží modelovať vzťah medzi dvoma
alebo viacerými vysvetľujúcimi premennými a závislou premennou vhodnou
lineárnou rovnicou pre pozorované dáta. Výsledný model je vyjadrený ako funkcia
viacerých vysvetľujúcich premenných~\cite{Grmanova2016}.

Túto funkciu môžeme zapísať ako
\begin{equation}
  Y(t) = V_t a_t + e_t
  \label{eq-multi-regresia}
\end{equation}
Vo vzorci~\ref{eq-multi-regresia} $t$ označuje čas, kedy bolo meranie
uskutočnené. $Y(t)$ predstavuje celkový nameraný odber elektrickej energie.
Vektor $V_t$ reprezentuje hodnoty vysvetľujúcich premenných v čase merania.
Vysvetľujúce premenné môžu predstavovať meteorologické vplyvy, ekonomický
nárast, ceny elektrickej energie či kurzy mien. Chybu modelu v čase $t$
zapíšeme ako $e_t$~\cite{KumarSingh2013, Mahalakshmi2016}.

%-------------------------------------------------------------------------------
%   Stochastic models
%-------------------------------------------------------------------------------

\subsubsection{Stochastické modely}
Tieto metódy časových radov sú založené na predpoklade, že dáta majú vnútornú
štruktúru, ako napr. autokoreláciu, trend či sezónnu varianciu. Najprv sa
precízne zostaví vzor zodpovedajúci dostupným dátam a potom sa na jeho základe
predpovie budúca hodnota veličiny~\cite{KumarSingh2013}.

\paragraph{Autoregresný model} predpovedá budúcu hodnotu premennej ako súčet
lineárnej kombinácie $p$ predchádzajúcich meraní, náhodnej chyby a konštanty.
V literatúre sa označuje ako AR (autoregressive model). Matematicky môžeme
autoregresný model zapísať ako
\begin{equation}
  y_t = c + \sum_{i=1}^{p} \varphi_i y_{t-i} + \varepsilon_t
  \label{eq-ar}
\end{equation}
Vo vzorci~\ref{eq-ar} hodnota $y_t$ predstavuje predpovedanú hodnotu
v čase $t$. Náhodnú chybu v čase $t$ zapíšeme ako $\varepsilon_t$. Hodnoty
$\varphi_i$ sú parametre modelu a $c$ je konštanta. Konštantou $p$ označujeme
rad modelu~\cite{Agrawal2013}.

\paragraph{Model kĺzavého priemeru} na rozdiel od autoregresného modelu používa
ako vysvetľujúce premenné chyby predchádzajúcich meraní a nie priamo hodnoty.
V literatúre sa označuje ako MA (moving average). Matematicky môžeme tento
vzťah zapísať ako
\begin{equation}
  y_t = \mu + \sum_{j=1}^{q} \Theta_j \varepsilon_{t-j} + \varepsilon_t
  \label{eq-ma}
\end{equation}
Vo vzorci~\ref{eq-ma} hodnota $y_t$ predstavuje strednú hodnotu
postupnosti meraní v čase $t$. Hodnoty $\Theta_j$ sú parametre modelu
a konštantou $q$ označujeme rad modelu. Vychádzame z predpokladu, že náhodná
zložka $\varepsilon_t$ je biely šum, čo je rovnomerne distribuovaná náhodná
premenná, ktorá má nulovú strednú hodnotu a konštantnú varianciu
$\sigma^2$~\cite{Agrawal2013}.

\paragraph{Autoregresívny model kĺzavého priemeru} reprezentuje súčasnú hodnotu
časového rádu lineárne na základe jeho hodnôt a hodnôt bieleho šumu
v predchádzajúcich periódach. V literatúre sa označuje ako ARMA (autoregressive
moving average)~\cite{KumarSingh2013}.

Ide o kombináciu autoregresie (AR) a kĺzavého priemeru (MA), vhodnú pre
modelovanie jednorozmerných časových radov. Matematicky môžeme reprezentovať
tento model ako súčet predchádzajúcich modelov
\begin{equation}
  y_t = c + \varepsilon_t + \sum_{i=1}^{p} \varphi_i y_{t-i}  \sum_{j=1}^{q} \Theta_j \varepsilon_{t-j}
  \label{eq-arma}
\end{equation}
Rad modelu určuje $p$ a $q$~\cite{Agrawal2013}.

\paragraph{Autoregresívny integrovaný model kĺzavého priemeru} je
generalizáciou modelu ARMA. V literatúre sa označuje ako ARIMA (autoregressive
integrated moving average). Modely typu ARMA môžu byť použité iba na statické
časové rady. Mnoho časových radov v praxi vykazuje nestatické správanie a napr.
tie, ktoré obsahujú komponenty trendu a sezónnosti. Kvôli tomu bol navrhnutý
model ARIMA, ktorý zahŕňa v sebe aj prípady nestatických časových radov.
Z nestatických časových radov sa vytvárajú statické pomocou konečného počtu
derivovaní dátových bodov. Vzniká tak matematický model, ktorý môžeme zapísať
ako
\begin{equation}
  \Big( 1 - \sum_{i=1}^{p} \varphi_i L^i \Big) (1-L)^d y_t = \Big( 1 + \sum_{j=1}^{q} \Theta_j L^j \Big) + \varepsilon_t
  \label{eq-arima}
\end{equation}
Vzorec~\ref{eq-arima} môžeme zapísať aj jednoduchšie a to
\begin{equation}
  \varphi(L) (1-L)^d y_t = \Theta(L) \varepsilon_t
  \label{eq-arima-short}
\end{equation}
Vo vzorci~\ref{eq-arima} predstavujú premenné $p$, $d$ a $q$ rad autoregresného
modelu, modelu kĺzavého priemeru a integrovaného modelu. Hodnota $d$ zodpovedá
stupňu derivovania, zvyčajne je rovná 1. V prípade, že $d=0$ dostaneme klasický
ARMA model. Rovnakým spôsobom vieme dostať modely AR a MA~\cite{Agrawal2013}.

%-------------------------------------------------------------------------------
%   Support vector regression
%-------------------------------------------------------------------------------

\subsubsection{Regresia založená na podporných vektoroch}
Regresia založená na podporných vektoroch a metóda podporných vektorov je
založená na štatistickej teórií učenia, nazývanej aj VC teória, podľa svojich
autorov, Vapnik a Chervonenkisa. Táto predikčná metóda poskytuje malý počet
voľných parametrov. Garantuje konvergenciu k ideálnemu riešeniu a môže byť
výpočtovo efektívna~\cite{Sapankevych2009}.

Metóda podporných vektorov je použitá na množstvo úloh strojového učenia ako je
rozoznávanie vzorov, klasifikácia objektov a v prípade predikcií časových
radov to je regresná analýza. Regresia založená na podporných vektoroch je
postup, ktorého funkcia je predpovedaná pomocou nameraných dát, ktorými je
metóda podporných vektorov natrénovaná. Toto je odklon od tradičných predpovedí
časových radov v zmysle, že metóda podporných vektorov nepoužíva žiadny model,
ale predikciu riadia samotné dáta~\cite{Sapankevych2009}.

Uvažujme množinu trénovacích dát
${(x_1, y_1), (x_2, y_2), ..., (x_n, y_n)} \subset \chi \times \mathbb{R}$, kde
v našom prípade hodnota $x_i$ predstavuje odber elektrickej energie v čase $i$
a množina $\chi$ označuje vstupnú množinu dát. Úlohou algoritmu je nájsť
funkciu  $f(x_i)$, pre ktorú hodnota $\varepsilon$ nadobúda pre všetky
trénovacie dáta čo najväčšie hodnoty oproti $y_i$ a súčasne nadobúda
najmenšie možné hodnoty menšie ako $\varepsilon$~\cite{Smola2004}.

Matematicky môžeme túto lineárnu zapísať ako
\begin{equation}
  f(x) = \langle w, x \rangle +  b
  \label{eq-svr}
\end{equation}
Vo vzorci~\ref{eq-svr} platí $w \in \chi$ a $b \in \mathbb{R}$. Výsledok
vektorového súčinu $\langle w, x \rangle$ sa nachádza v množine $\chi$. Našou
snahou je dosiahnuť čo najmenšiu hodnotu $w$. Zabezpčiť to môžeme
minimalizovaním hodnoty $|| w ||^2 = \langle w, w \rangle$. Pri cieli
minimalizovať hodnotu $\frac{1}{2} || w ||^2$, môžeme opisovaný problém zapísať
ako
\begin{equation}
  \begin{cases}
    y_i - \langle w, x_i \rangle - b \leq \varepsilon \\
    \langle w, x_i \rangle + b - y_i \leq \varepsilon \\
  \end{cases}
  \label{eq-svr-min}
\end{equation}
Predpokladom pre vzorec~\ref{eq-svr-min} je, že funkcia $f$ existuje pre všetky
páry $(x_i, y_i)$ s presnosťou~$\varepsilon$~\cite{Smola2004}.

%-------------------------------------------------------------------------------
%   Decision trees
%-------------------------------------------------------------------------------

\subsubsection{Rozhodovacie stromy}
Rozhodovacie stromy sú jednou z najrozšírenejších učiacich metód. Používajú sa
najmä na klasifikáciu, ale v súčasnosti sa využívajú aj na regresiu.
Rozhodovací strom je reprezentovaný ako množina uzlov a im prislúchajúcich
hrán. Uzly reprezentujú atribúty a výstupné hrany sú vždy označené konkrétnou
hodnotou pre atribút, z ktorého vychádzajú. Rozhodovanie začína v koreni stromu
a končí po dosiahnutí listového uzla. Pre riešenie jedného problému je možné
vytvoriť stromy s rôznym počtom a usporiadaním uzlov. Najlepším riešením je
strom s najmenším počtom rozhodovacích uzlov~\cite{Merz1998}.

Hlavnou výhodou rozhodovacích stromov oproti ostatným modelom je, že strom
produkuje model, ktorý je reprezentovateľný ako pravidlá alebo logické výroky.
Taktiež klasifikácia sa môže vykonávať bez komplikovaných výpočtov. Táto
technika môže byť použitá ako na kategorické premenné tak aj na spojité.
Rozhodovacie stromy neposkytujú také dobré výsledky pre nelineárne dáta ako
neurónové siete. Vo všeobecnosti je táto metóda vhodnejšia pre predpovedanie
kategorických dát alebo na dáta, ktoré v sebe obsahujú viditeľný
trend~\cite{Tso2007}.

%-------------------------------------------------------------------------------
%   Regression trees
%-------------------------------------------------------------------------------

\paragraph{Regresný rozhodovací strom} obsahuje odozvový vektor $Y$
reprezentujúci odozvu hodnôt ku každému meraniu v matici $X$. Vetvy sú
rozdeľované na základe štvorcového zostatkového minimalizačného algoritmu.
Ten zabezpečuje, že očakávaný súčet variancií dvoch uzlov bude minimalizovaný.
Algoritmus tak nájde optimálnu podmienku rozdelenia uzlu na jeho
potomkov~\cite{Bel2009}.

Priradením hodnoty 1 pre triedu $k$ a hodnoty 0 pre ostatné triedy, získame
varianciu rovnú $p(k|t)[1 - p(k|t)]$. Sčítaním $K$ tried získame funkciu
\begin{equation}
  i(t) = 1 - \sum_{k = 1}^K p^2 (k|t)
  \label{eq-tree-impurity}
\end{equation}
Tým sme vytvorili maximálny strom, čo znamená, že uzly sa rozdeľovali až do
posledného merania, ktoré sa nachádzalo v trénovacej množine. Maximálny strom
sa tak môže stať veľmi veľkým~\cite{Bel2009}.

Jedným z najpoužívanejších algoritmov na rozdeľovanie uzlov sa nazýva index
Gini alebo rozhodovacie pravidlo Gini. Tento algoritmus zapíšeme ako
\begin{equation}
  i(t) = \sum_{k \neq l} p(k|t) p(l|t)
  \label{eq-tree-gini}
\end{equation}
Vo vzorci~\ref{eq-tree-gini} predstavujú $k$ a $l$ indexy tried o 1 po $K$ a
$p(k|t)$ označuje podmienenú pravdepodobnosť triedy $k$ za predpokladu, že
aktuálny uzol je uzol $t$. Aplikovaním Gini algoritmu nájdeme v trénovacej
množine najväčšiu triedu, ktorú odizolujeme od ostatných dát~\cite{Bel2009}.

%-------------------------------------------------------------------------------
%   Random forest
%-------------------------------------------------------------------------------

\paragraph{Náhodné lesy} sú kombináciou predpovedí stromov. Každý strom závisí
od hodnoty náhodného vektora s rovnakým rozdelením. Chyba lesu závisí od sily
jednotlivých strom a koreláciou medzi nimi. Náhodný les môžeme definovať ako
klasifikátor pozostávajúci z množiny stromov $\{h(x, \Theta_k), k=1, 2, ... \}$,
kde $\{\Theta_k\}$ sú nezávislé rovnomerne distribuované náhodné vektory a každý
strom sa podieľa hlasom na voľbe triedy vstupu $x$. S nárastom počtu stromov
hodnota $\{\Theta_k\}$ konverguje k určitému bodu. Tým je zabezpečené, že
náhodné lesy sa s pridávajúcim počtom stromov nepretrénujú, ale veľkosť chyby
sa postupne ustáli. Pri výbere náhodného vektora sa snažíme pri zachovaní jeho
sily minimalizovať koreláciu, čím zvyšujeme presnosť celého
výpočtu~\cite{Breiman2001}.

Väčšinou sú náhodné lesy používané na klasifikačné problémy, avšak je možné ich
aplikovať aj na regresiu. Regresné náhodné lesy sú tvorené rastom stromov
závislých na náhodnom vektore $\{\Theta\}$. Prediktor stromu $h(x, \Theta)$
nadobúda číselné hodnoty na rozdiel od štítkov tried ako je to pri klasifikačných
problémoch. Predpokladáme trénovaciu množinu, ktorá je nezávislou distribúciou
náhodného vektora $Y, X$. Potom môžeme strednú štvorcovú generalizačnú chybu
pre číselný prediktor $h$ matematicky zapísať ako
\begin{equation}
  E_{X, Y} (Y - h(X))^2
  \label{eq-random-error}
\end{equation}
Prediktor náhodného lesu je tvorený priemerom $k$ stromov, čo zapíšeme ako
$h(x, \Theta_k)$~\cite{Breiman2001}.

%-------------------------------------------------------------------------------
%   Neural networks
%-------------------------------------------------------------------------------

\subsubsection{Neurónové siete}
Návrh neurónových sietí je inšpirovaný neurofyziológiou ľudského mozgu. Model
je analytická technika modelujúca procesy učenia v kognitívnych systémoch
a neurologických funkciách mozgu. Má schopnosť predpovedať budúcu hodnotu
merania konkrétnej premennej na základe hodnôt z predchádzajúcich meraní. Tento
proces sa inak nazýva aj učenie z existujúcich dát. Tok v neurónovej sieti
preteká cez jednotlivé neuróny. Na obrázku~\ref{fig-neuron} môžeme vidieť
príklad takéhoto neurónu~\cite{Tso2007}.

\begin{figure}[!ht]
  \centering
  \includegraphics[width=0.6\textwidth]{neuron.png}
  \caption{Príklad neurónu v neurónovej sieti~\cite{Tso2007}.}
  \label{fig-neuron}
\end{figure}

Neurónová sieť predstavuje orientovaný graf uzlov. Uzol neurónovej
siete sa nazýva neurón. Každý uzol počíta svoj výstup na základe vstupov od
susedných uzlov. Výpočet prebieha aplikovaním funkcie, ktorá sa nazýva sigmoid,
na vážený súčet vstupov~\cite{Gruau1994}.

Trénovanie siete je proces nastavovania, čo najlepších váh na vstupy
jednotlivých neurónov. Chyba neurónovej siete sa najčastejšie počíta pomocou
spätnej propagácie (backpropagation), čím dostaneme rast chyby pre danú
neurónovú sieť~\cite{Tso2007}.

Je veľa typov neurónových sietí napríklad viacvrstvové perceptónové siete,
samoriadiace siete, siete s viacerými skrytými vrstvami, alebo aj viacvrstvové
spätne propagované neurónové siete, pričom príklad takejto siete môžeme vidieť
aj na obrázku~\ref{fig-neural-network}. V každej skrytej vrstve je množstvo
neurónov. Hlavnou výhodou je, že väčšina sietí nepotrebuje model. Na druhej strane,
trénovanie obvykle zaberá veľa času. Výstupom siete je lineárna rovnica váh
prepojených so vstupom~\cite{KumarSingh2013}.

\begin{figure}[!ht]
  \centering
  \includegraphics[width=\textwidth]{neural_network.png}
  \caption{Príklad viacvrstvovej spätne propagovanej neurónovej siete~\cite{Kennedy2002}.}
  \label{fig-neural-network}
\end{figure}

\textbf{Viacvrstvový perceptón} je jednou z najpoužívanejších neurónových
sietí. Pozostáva z uzlov a im prislúchajúcim hranám. Uzly sú zoskupované do
rôznych vrstiev. Prvá vrstva je vstupná vrstva, kde počet $d$ označuje počet
vstupných parametrov vstupujúcich do siete. Táto vrstva je následne prepojená
hranami so skrytou vrstvou pozostávajúcou z $h$ uzlov. Tá je potom prepojená
s výstupnou vrstvou s $c$ uzlami. Kvôli tomu sa tieto siete zvyknú označovať aj
ako dopredné siete~\cite{Merz1998}.

Elementy skrytých a výstupných vrstiev sú umelé neuróny pozostávajúce z uzlov,
viacerých vstupujúcich a jednou výstupnou hranou. Funkciou neurónu je
transformovať lineárnu kombináciu vstupov pomocou nelineárnej aktivačnej
funkcie, čiže každú vstupnú hranu prenásobiť jej váhou a výsledok týchto súčinov
sčítať. Tak dostaneme pre neurón $j$ vzorec~\ref{eq-linear-combination}
opisujúci lineárnu kombináciu vstupov $a_j$
\begin{equation}
  a_j = \sum_{i=1}^{d} w_{ji} x_i
  \label{eq-linear-combination}
\end{equation}
Pričom váha $w_{ji}$ označuje váhu medzi neurónom $i$ na vstupnej vrstve
a neurónom $j$ na skrytej vrstve~\cite{Merz1998}.

Aktivovanie neurónu $j$ závisí od jeho aktivačnej funkcie $g(a_j)$. Jednu
z najpoužívanejších aktivačných funkcií, logistickú sigmoidnú funkciu, môžeme
matematicky zapísať ako
\begin{equation}
  g(a) \equiv \frac{1}{1 + exp(-a)}
  \label{eq-sigmoid-function}
\end{equation}
Je zrejmé, že funkcia zo vzorca~\ref{eq-sigmoid-function} vracia hodnoty
v rozmedzí $(0,1)$~\cite{Merz1998}.

%-------------------------------------------------------------------------------
%   Analysis of optimizing algorithms
%-------------------------------------------------------------------------------

\subsection{Analýza optimalizačných algoritmov}
Optimalizačné algoritmy, ktoré majú potenciál nájsť globálne alebo lokálne
riešenie problému. Lokálne optimalizácie, nazývané aj vyhľadávacie algoritmy,
sa pokúšajú nájsť lokálne minimum v okolí štartovacieho riešenia. Väčšina
týchto algoritmov je deterministická. Pri hľadaní minima používajú
vyhodnocovaciu funkciu, na základe ktorej aktualizujú doterajšie riešene.
Z nových možných riešení je najlepšie to s najnižšou hodnotou, čiže to, ktoré
je najlacnejšie. Kvôli tejto vlastnosti sa zvyknú tieto algoritmy označovať aj
ako nenásytné algoritmy (greedy algorithms). Spravidla tieto algoritmy nenájdu
globálne minimum v prípade, že sa nachádza ďalej od štartovacieho riešenia ako
nejaké lokálne minimum~\cite{Sen1995}.

Na druhej strane, optimalizačné algoritmy s potenciálom nájdenia globálneho
minima nachádzajú riešenie, ktoré postupne konverguje k optimálnemu riešeniu.
To ale nie je algoritmami úplne garantované. Algoritmy s globálnym
potenciálom majú väčší prehľad o svojom okolí, a preto uviaznutie v lokálnom
minime je zriedkavé~\cite{Sen1995}.

Nasledujúca tabuľka~\ref{tab-bio} zobrazuje rozdelenie algoritmov
s potenciálom nájdenia globálneho minima. Algoritmy sú prírodne inšpirované
a ich ďalšie delenie vyplýva zo spoločných znakov, na ktorých sú založené.

\begin{table}
  \centering
  \caption{Tabuľka rozdelenia vybraných prírodne inšpirovaných algoritmov~\cite{Goel2012}.}
  \label{tab-bio}
  % \begin{tabular}{|p{34mm}|p{34mm}|p{34mm}|p{34mm}|}
  \begin{tabular}{|l|l|l|l|}
    \hline
    \multicolumn{1}{|c|}{\textbf{\begin{tabular}[c]{@{}c@{}}Model ľudskej \\ mysle\end{tabular}}} & \multicolumn{1}{c|}{\textbf{\begin{tabular}[c]{@{}c@{}}Umelé imunitné \\ systémy\end{tabular}}} & \multicolumn{1}{c|}{\textbf{\begin{tabular}[c]{@{}c@{}}Rojová \\ inteligencia\end{tabular}}} & \multicolumn{1}{c|}{\textbf{\begin{tabular}[c]{@{}c@{}}Ostatné napr. \\ prírodné úkazy\end{tabular}}} \\ \hline
    Teória fuzzy množín                                                                           & Genetický algoritmus                                                                            & \begin{tabular}[c]{@{}l@{}}Optimalizácia \\ kolóniu mravcov\end{tabular}                     & Tektonické dosky                                                                                      \\
    \textbf{Rough Set theory}                                                                     & Umelé neurónové siete                                                                           & \begin{tabular}[c]{@{}l@{}}Optimalizácia \\ rojom častíc\end{tabular}                        & Veľký kolaps                                                                                          \\
    \textbf{Granular COmputing}                                                                   & Celulárný automat                                                                               & \textbf{\begin{tabular}[c]{@{}l@{}}Foraging \\ bacteria\end{tabular}}                        & Oceánske prúdy                                                                                        \\
    \textbf{\begin{tabular}[c]{@{}l@{}}Perception-Based \\ Computing\end{tabular}}                & \textbf{\begin{tabular}[c]{@{}l@{}}Membrane \\ computing\end{tabular}}                          & \begin{tabular}[c]{@{}l@{}}Včelí zhlukovací \\ algoritmus\end{tabular}                       & Prílivové vlny                                                                                        \\
    \textbf{Wisdom Technology}                                                                    & \textbf{DNA computing}                                                                          & \begin{tabular}[c]{@{}l@{}}Vyhľadávanie \\ kukučkou\end{tabular}                             & Sopečné erupcie                                                                                       \\
    \textbf{\begin{tabular}[c]{@{}l@{}}Anticipatory \\ Computing\end{tabular}}                    &                                                                                                 & \begin{tabular}[c]{@{}l@{}}Inteligentná \\ kvapka vody\end{tabular}                          & Zemetrasenia                                                                                          \\
    \hline
  \end{tabular}
\end{table}

%-------------------------------------------------------------------------------
%   Genetic algorithms
%-------------------------------------------------------------------------------

\subsubsection{Genetický algoritmus}
Genetický algoritmus patrí medzi prírodne inšpirované algoritmy patriace do
triedy umelých imunitných systémov. Genetický algoritmus je stochastický
optimalizačný algoritmus, ktorého úlohou je nájdenie globálneho riešenia pre
zadaný problém, čiže sa nestane, že riešenie spadne do lokálne minima a nenájde
sa tak optimálne riešenie. Od tradičných algoritmov sa líšia hlavne v počte
riešení, ktoré sú kandidátmi na najlepšie riešenie. Tradičné vyhľadávacie
algoritmy prehľadávajú dôkladne iba jedno riešenie, zatiaľ čo genetické
algoritmy hlbšie spracujú viacero kandidátov naraz. Každý kandidát na optimálne
riešenie problému je reprezentovaný dátovou štruktúrou, ktorú označujeme pojmom
jedinec. Súbor jedincov tvorí populáciu. Začiatok procesu začína náhodnými
riešeniami populácie, ktorý sa postupne vylepšuje~\cite{Chavan2015}.

Vytvorenie novej generácie sa vykonáva pomocou genetických operátorov:
a to selekciou, krížením a mutáciou. Proces selekcie vyberie kvalitnejšie
chromozómy, ktoré prežijú a vyskytnú sa tak aj v ďalšej
generácii~\cite{Simonova2007}.

Pri genetických algoritmoch sa zavádzajú pojmy ako chromozóm, fitness funkcia,
kríženie, elitárstvo, operátor reprodukcie či mutácie~\cite{Chavan2015}.

\paragraph{Chromozóm} je pomenovanie pre jedinca. V literatúre sa tiež zvykne
používať označenie kandidát~\cite{Arun2016}.

\paragraph{Inicializácia} je proces, ktorý po vygenerovaní generácie priradí
kandidátom náhodné hodnoty. Pri prehľadávaní dvojrozmerného priestoru to budú
náhodné hodnoty oboch súradníc~\cite{Lazinica2009}.

\paragraph{Fitness funkcia} je funkcia určujúca efektívnosť chromozómu. Každý
jedinec v množine je ohodnotený pomocou tejto funkcie, ktorá vracia číselnú
reprezentáciu riešenia, čiže ako dobre daný kandidát vyriešil zadaný problém.
Pri hľadaní optimálneho riešenia sa porovnáva hodnota fitness funkcie aktuálneho
riešenie s hodnotou funkcie cieľového riešenia, ale vo všeobecnosti, čím je
hodnota väčšia, tým je kandidátovo riešenie
lepšie~\cite{Chavan2015, Lazinica2009, Simonova2007}.

\paragraph{Operátor reprodukcie} je obvykle prvý operátor, ktorý sa uplatní na
populáciu. Operátor náhodne vyberie reťazce z dvoch chromozómov na
párenie~\cite{Chavan2015}.

\paragraph{Kríženie} je operátor kombinácie. Kríženie vykonáva výmenu blokov
chromozómov. Z druhého chromozómu je vybraný reťazec náhodnej veľkosti, ktorý
sa vymení s rovnako dlhým reťazcom z prvého chromozómu.
V tabuľke~\ref{tab-crossing} je tento proces znázornený
graficky~\cite{Chavan2015}.

\begin{table}[H]
  \centering
  \caption{Proces vytvorenia novej generácie z rodičovských chromozómov.}
  \label{tab-crossing}
  \begin{tabular}{p{4cm}l|l|l}
    rodič č. 1      &  0110 & 0101  0010 & 0011  \\
    rodič č. 2      &  1100 & 1101  1110 & 1111  \\
    \hline
    potomok č. 1    &  0110 & 1101  1110 & 0011  \\
    potomok č. 2    &  1100 & 0101  0010 & 1111  \\
  \end{tabular}
\end{table}

Vzniká problém ako čo najvhodnejšie určiť bod kríženia a vybrať tak
najkvalitnejšie bloky chromozómu. Kvôli tomu je potrebné definovať nový element
nazývaný väzba, označovaný ako $d_i$
\begin{equation}
  d_i = \frac{a_i + a_{i+1}}{2}
  \label{eq-crossing}
\end{equation}
Za predpokladu, že chromozóm reprezentujeme ako maticu veľkosti $1 \times N$,
potom väzbu definovanú vzorcom \ref{eq-crossing} môžeme reprezentovať ako
maticu $1 \times (N-1)$. Vypočítaním priemeru susedných hodnôt aj pre cieľovú
maticu a nájdením priemeru matice, určíme body kríženia. Chromozóm rozdelíme na
miestach, kde je väzba, čo najmenšia~\cite{Simonova2007}.

\paragraph{Elitárstvo} je proces pridávania chromozómov s najlepšou hodnotou
funkcie fitness priamo do ďalšej populácie. Zaisťuje to, že najlepšie riešenie
budúcej generácie bude vždy lepšie alebo pri najhoršom rovnaké, ako najlepšie
riešenie predchádzajúcej generácie~\cite{Deolekar2016}.

\paragraph{Operátor mutácie} sa vykoná po vykonaní operátora reprodukcie.
Mutácia chromozómu väčšinou predstavuje operáciu, ktorá s nízkou
pravdepodobnosťou invertuje jeden bit v chromozóme~\cite{Chavan2015}.

\paragraph{Princíp genetického algoritmu} môžeme znázorniť v nasledujúcom
pseudokóde~\cite{Chavan2015}
\begin{algorithm}
  \caption{Pseudokód genetického algoritmu}
  \begin{algorithmic}[1]
    \State Náhodne inicializovanie jedincov
    \State Vyhodnotenie fitness funkcie pre každého jedinca \label{ga-fitness}
    \State Výber jedincov pre ďalšiu populáciu na základe fitness funkcie
    \State Kríženie jedincov
    \State Mutovanie jedincov
    \State Ak bolo nájdené žiadané optimálne riešenie pokračuj, inak návrat na krok \ref{ga-fitness}
    \State Vráť optimálne riešenie
  \end{algorithmic}
\end{algorithm}

Veľkou výhodou genetického algoritmu je, že mutácia predchádza skĺznutiu do
lokálnych miním a kombinácia chromozómov vedie k rýchlemu približovaniu
sa k optimálnemu riešeniu. Napriek týmto výhodám, majú genetické algoritmy aj
niekoľko nevýhod~\cite{Deolekar2016}.
\begin{itemize}
  \item Reprezentovanie kandidátov je príliš obmedzujúce
  \item Mutácia a kríženie sú v súčasnosti aplikovateľné iba na chromozómy
        reprezentované bitovým reťazcom alebo číslami
  \item Definovanie fitness funkcie je často netriviálnou záležitosťou
        a jej generalizácia je náročná
\end{itemize}

%-------------------------------------------------------------------------------
%   Grey wolf optimizer
%-------------------------------------------------------------------------------

\subsubsection{Optimalizácia svorkou divých vlkov}
Algoritmus je založený na správaní vlka sivého, ktorý je na vrchole
potravinového reťazca a preferuje život vo svorke. Lov pozostáva zo stopovania,
prenasledovania, približovania sa, obkľúčenia koristi a útokom na korisť. Vlkov
vo svorke možno rozdeliť do niekoľkých skupín.
V obrázku~\ref{fig-wolf-hierarchy} je znázornená hierarchia týchto
skupín~\cite{Seeley1991}.

\begin{figure}[!ht]
  \centering
  \includegraphics[width=0.5\textwidth]{wolf_hierarchy.png}
  \caption{Hierarchia vlkov vo svorke~\cite{Seeley1991}.}
  \label{fig-wolf-hierarchy}
\end{figure}

\paragraph{Alfa vlky} sú vodcami svorky. Ich úlohou je robiť rozhodnutia
ohľadom lovu, miesta na spanie či času zobudenia. Tieto príkazy diktujú svorke,
avšak bolo pozorované aj demokratické správanie, kedy alfa vlky nasledovali
ostatných členov svorky. Všetky vlky uznávajú postavenie alfy vo svorke.
Zaujímavosťou je, že vodcom nemusí byť najsilnejší jedinec, ale môže to
byť aj jedinec najlepší v organizovaní svorky~\cite{Seeley1991}.

\paragraph{Beta vlky} sú druhým stupňom v hierarchii, podriadené alfa vlkom,
ktorým pomáhajú v rozhodovaní a organizácií. Sú najlepšími kandidátmi na alfu
v prípade úmrtia alebo zostarnutia alfa jedincov. Hrajú rolu radcu a ďalej
distribuujú príkazy a vracajú sa s odozvou na ne~\cite{Seeley1991}.

\paragraph{Delta vlky} inak nazývané aj podriadené, zastupujú vo svorke
úlohy prieskumníkov, strážcov, starejších, lovcov či opatrovateľov.
Prieskumníci hliadkujú hranice a varujú svorku pred nebezpečím. Strážcovia
sa starajú o bezpečie svorky. Starejší sú skúsenými vlkmi, ktorí boli na pozícii
alfa alebo beta. Lovci pomáhajú alfa a beta vlkom pri love. Opatrovatelia sa
starajú o slabé, choré alebo zranené jedince~\cite{Seeley1991}.

\paragraph{Omega vlky} majú najnižšiu hodnosť vo svorke. Hrajú rolu obetných
baránkov, ktoré sa podrobia ostatným. K potrave sa dostanú ako úplne posledné.
Aj keď sa možno javí ich postavenie zbytočné, boli pozorované prípady, kedy ich
strata spôsobila vo svorke nedorozumenia~\cite{Seeley1991}.

\paragraph{Spoločenská hierarchia} reprezentovaná matematickým modelom,
označuje najvhodnejšie riešenie ako alfa, druhé a tretie najvhodnejšie ako beta
resp. delta. Riešenia ostatných kandidátov označujeme ako omega. Algoritmus
optimalizácie (v prírode lovu) je vedený alfa, beta a delta kandidátmi, ktorí
sú nasledovaní kandidátmi omega~\cite{Seeley1991}.

\paragraph{Obkľúčenie koristi} jednotlivými vlkmi $\alpha$, $\beta$ a $\delta$
môžeme matematicky vyjadriť nasledujúcimi rovnicami
\begin{equation}
  \vec{D_\alpha} = | \vec{C_1} \cdot \vec{X_\alpha} - \vec{X} |
  \label{eq-prey-alpha}
\end{equation}

\begin{equation}
  \vec{X_1} = \vec{X_\alpha} - \vec{A_1} \cdot \vec{D_\alpha}
  \label{eq-prey-x1}
\end{equation}

\begin{equation}
  \vec{D_\beta} = | \vec{C_2} \cdot \vec{X_\beta} - \vec{X} |
  \label{eq-prey-beta}
\end{equation}

\begin{equation}
  \vec{X_2} = \vec{X_\beta} - \vec{A_2} \cdot \vec{D_\beta}
  \label{eq-prey-x2}
\end{equation}

\begin{equation}
  \vec{D_\delta} = | \vec{C_3} \cdot \vec{X_\delta} - \vec{X} |
  \label{eq-prey-delta}
\end{equation}

\begin{equation}
  \vec{X_3} = \vec{X_\delta} - \vec{A_3} \cdot \vec{D_\delta}
  \label{eq-prey-x3}
\end{equation}
Vo vzorcoch~\ref{eq-prey-x1},~\ref{eq-prey-x2}~a~\ref{eq-prey-x3} predstavujú
vektory $\vec{X_1}$, $\vec{X_2}$ a $\vec{X_3}$ polohu koristi a vektory
$\vec{X_\alpha}$, $\vec{X_\beta}$ a $\vec{X_\delta}$ polohu vlkov
$\alpha$, $\beta$ a $\delta$. Vektory $\vec{A}$ a $\vec{C}$ sú koeficienty,
ktoré zapíšeme ako
\begin{equation}
  \vec{A} = 2\vec{a} \cdot \vec{r_1} - \vec{a}
  \label{eq-prey-a}
\end{equation}

\begin{equation}
  \vec{C} = 2 \cdot \vec{r_2}
  \label{eq-prey-c}
\end{equation}
Vo vzorci~\ref{eq-prey-a} sa premenná $\vec{a}$ počas výpočtu lineárne
znižuje od 2 po 0. Vektory $\vec{r_1}$ a $\vec{r_2}$ vo vzorci~\ref{eq-prey-c}
sú náhodnými vektormi v rozsahu [0, 1]. Vlk môže svoju pozíciu $(X, Y)$
aktualizovať v závislosti od pozície koristi $(X^*, Y^*)$.
Obrázok~\ref{fig-wolf-pos} ilustruje možné aktualizované pozície, ktoré môže
najlepší agent dosiahnuť. Tieto pozície získame aplikovaním
vzorcov~\ref{eq-prey-alpha}~až~\ref{eq-prey-x3}~\cite{Seeley1991}.

\begin{figure}[ht]
  \centering
  \includegraphics[width=\textwidth]{wolf_vector_positions.png}
  \caption{Príklad možného umiestnenia vlka v dvojrozmernom priestore na základe polohy koristi~\cite{Seeley1991}.}
  \label{fig-wolf-pos}
\end{figure}

\paragraph{Hľadanie koristi} je zabezpečené vektorom $\vec{A}$, ktorý mimo
intervalu [-1, 1] núti vlky divergovať od seba, čím je zdôraznená potreba
prehľadávať okolie a neskĺznuť do lokálneho minima. Náhodný vektor $\vec{C}$
simuluje prekážky v prírode, s ktorými sa vlk stretne pri hľadaní koristi.
V závislosti od vygenerovanej hodnoty, môže simulovať aj opačný prípad, ktorý
je pre vlka priaznivejší~\cite{Seeley1991}.

\paragraph{Lov} je vedený alfa vlkmi, beta a delta vlky sa tiež môžu na ňom
príležitostne podieľať. V prírode disponujú schopnosťou rozpoznať
umiestnenie koristi a obkľúčiť ju, avšak pri simulácií tohto správania, nemáme
vedomosť o presnej polohe koristi. Preto na základe prvých troch najlepších
riešení vypočítame predpokladanú polohu koristi, čím vznikne
vzorec~\ref{eq-prey-pos}. Kandidáti, vrátane omega vlkov, potom na základe
rovníc~\ref{eq-prey-x1},~\ref{eq-prey-x2}~a~\ref{eq-prey-x3} aktualizujú svoju
polohu~\cite{Seeley1991}.
\begin{equation}
  \vec{X}(t+1) = \frac{\vec{X_1} + \vec{X_2} + \vec{X_3}}{3}
  \label{eq-prey-pos}
\end{equation}

\paragraph{Útok} na korisť v matematickom modely dosiahneme približovaním sa ku
koristi, čiže znížením hodnoty $\vec{a}$. Tak dosiahneme aj zníženie hodnoty
$\vec{A}$ až bude jeho hodnota v intervale [-1, 1]. Potom ďalšia vypočítaná
hodnota pozície vlka sa bude nachádzať medzi jeho pôvodnou polohou a polohou
koristi. V prípade, že hodnota $\vec{A}$ sa nachádza mimo spomínaného
intervalu, vlk diverguje od koristi, čo nastáva vo fáze jej hľadania. Tým
je zabezpečené prehľadávanie priestoru agentami~\cite{Seeley1991}.

\paragraph{Princíp optimalizácie svorkou divých vlkov} môžeme znázorniť
v nasledujúcom pseudokóde~\cite{Seeley1991}
\begin{algorithm}[H]
  \caption{Pseudokód optimalizácie svorkou divých vlkov}
  \begin{algorithmic}[1]
    \State Náhodná inicializácia vlkov
    \State Inicializácia $a$, $A$ a $C$
    \State Vyhodnotenie fitness funkcie pre každého jedinca
    \State Priradenie do 3 najlepších riešení do $X_\alpha$, $X_\beta$ resp. $D_\delta$
    \State Opakuj kroky \ref{gwo-while} až \ref{gwo-iter} pokiaľ nebude prekročený maximálny počet iterácií \label{gwo-while}
    \State Aktualizovanie polohy na základe rovníc rovníc~\ref{eq-prey-x1},~\ref{eq-prey-x2}~a~\ref{eq-prey-x3}
    \State Aktualizovanie $a$, $A$ a $C$
    \State Vyhodnotenie fitness funkcie pre každého jedinca
    \State Aktualizovanie $X_\alpha$, $X_\beta$ a $D_\delta$
    \State Aktualizovanie počtu iterácií \label{gwo-iter}
    \State Vráť optimálne riešenie $X_\alpha$
  \end{algorithmic}
\end{algorithm}

%-------------------------------------------------------------------------------
%   Artifical bee colony
%-------------------------------------------------------------------------------

\subsubsection{Umelá kolónia včiel}
V literatúre označovaný ako ABC algoritmus (Artifical bee colony) je pomerne
nový medzi rojovými algoritmami. Princíp je založený na biologickom procese
správaní medonosných včiel pri hľadaní potravy. Každá včela pracujúca
v roji sa spolupodieľa na tvorbe celého systému na globálnej úrovni. Správanie
systému je určené lokálnym správaním, kde spolupráca a zladenie jedincov vedie
k štruktúrovanému kolaboračnému systému~\cite{Chavan2015}.

V algoritme, kolónia umelých včiel pozostáva z 3 skupín. Sú to včely
robotnice, diváčky a prieskúmníčky. Informácie o zdrojoch potravy si vymieňajú
tancovaním (waggle dance) na tzv. tanečnej ploche. Diváčky čakajúce na tanečnej
ploche sa rozhodujú, ktorý zdroj potravy si vyberú. Robotnice najskôr tieto
zdroje navštívia a prieskumníčky uskutočňujú náhodné prehľadávanie priestoru.
Pri aplikovaní ABC algoritmu, polovica včiel
predstavuje robotnice, druhá polovica diváčky. Počet robotníc zodpovedá počtu
zdrojov potravy v okolí úľu. Robotnica, ktorej zdroj je vyčerpaný inými
včelami sa stáva prieskumníčkou~\cite{Karaboga2007}.

Zväčšovaním množstva zdrojov potravy sa zväčšuje aj pravdepodobnosť vybratia
zdroja diváčkou. Vďaka tomu je zabezpečené, že tanec robotníc, ktoré navštívili
zdroje s najväčším množstvom nektáru, presvedčí najviac diváčok. Tie
na základe informácie o polohe zdroja nájdu v jeho okolí nový zdroj, z ktorého
budú zberať nektár. Po vyčerpaní nektáru je tento zdroj opustený a včela sa
presunie na zdroj, ktorý našli prieskumníčky~\cite{Karaboga2007}.

Pozícia zdrojov jedla je reprezentácia možného riešenia daného problému.
Množstvo nektáru je úmerné kvalite riešenia, ktoré je určené fitness funkciou.
Počet robotníc a diváčok je rovný počtu riešení v danej populácií, pričom každé
je reprezentovateľné ako D-dimenzionálny vektor. Hodnota $D$ môže
predstavovať počet optimalizačných parametrov~\cite{Karaboga2007}.

Včely majú riešenie uložené vo vlastnej pamäti, z ktorého modifikáciou vznikajú
nové, ktoré následné vyskúšajú. Ak je nové riešenie lepšie ako predchádzajúce,
včely si ho zapamätajú a predchádzajúce zabudnú. Inak si pamätajú staré. Keď
všetky robotnice skončia proces hľadania, zdieľajú informácie o zdrojoch
s diváčkami. Tie vyhodnotia zhromaždené informácie a vyberú zdroj s najlepším
riešením a jeho modifikovaním vytvoria nové, ktoré následne
skontrolujú~\cite{Karaboga2007}.

Diváčky vyberajú zdroj v závislosti od pravdepodobnosti zdroja $p_i$,
ktorú vypočítame ako
\begin{equation}
  p_i = \frac{fit_i}{\sum_{j = 1}^{n} fit_j}
  \label{eq-abc-prop}
\end{equation}
Vo vzorci \ref{eq-abc-prop} predstavuje $fit_i$ fitness funkciu, ktorá
ohodnocuje riešenie $i$. Týmto spôsobom si vymieňajú medzi sebou informácie
robotnice a diváčky~\cite{Karaboga2007}.

Algoritmus počíta nové riešenie na základe starého, čo matematicky zapíšeme ako
\begin{equation}
  v_{ij} = x_{ij} + \phi_{ij} (x_{ij} - x_{kj})
  \label{eq-abc-src}
\end{equation}
Ak si počet včiel definujeme ako $B$ potom na základe vzorca \ref{eq-abc-src}
platí $k \in {1, 2, ..., B}$ a $j \in {1, 2, ..., D}$. Indexy $k$ a $j$ sú
vybrané náhodne, ale $k$ musí byť rôzne od $i$. Hodnota $\phi_{ij}$ je
z intervalu [-1, 1] vyberaná náhodne. Je zrejmé, že čím bude menší rozdiel
medzi $(x_{ij}$ a $x_{kj})$, tým bude menší rozdiel aj medzi starým a novým
riešením~\cite{Karaboga2007}.

Algoritmus má 3 konfigurovateľné parametre a to: počet robotníc alebo diváčok,
ktorý je rovný počtu zdrojov jedla, limit, ktorý ohraničuje prehľadávaný
priestor a maximálny počet cyklov, ktorý sa vykoná ak sa skôr nenájde
optimálne riešenie~\cite{Karaboga2007}.

\paragraph{Princíp umelej kolónii včiel} znázorňuje nasledujúci
pseudokód~\cite{Karaboga2007}
\begin{algorithm}[H]
  \caption{Pseudokód umelej kolónie včiel}
  \begin{algorithmic}[1]
    \State Inicializácia včiel na náhodné zdroje potravy
    \State Opakuj kroky \ref{abc-while} až \ref{abc-end} pokiaľ nebude dosiahnutý cieľ \label{abc-while}
    \State Umiestnenie robotníc na zdroje potravy, ktoré boli nájdené
    \State Vyhodnotenie zdroja na základe množstva nektáru
    \State Umiestnenie diváčok na zdroje potravy na základe získaných informácií
    \State Vyhodnotenie, ktoré včely sa stanú prieskumníčkami
    \State Vyslanie prieskumníčok do prehľadávanie priestoru za účelom objavenia nových zdrojov potravy \label{abc-end}
    \State Vráť optimálne riešenie
  \end{algorithmic}
\end{algorithm}

%-------------------------------------------------------------------------------
%   Measurement of prediction accuracy
%-------------------------------------------------------------------------------

\subsection{Meranie presnosti predpovede}
Pre vyhodnotenie efektívnosti a presnosti modelov je potrebné merať ich
vlastnosti tak, aby sme ich vedeli medzi sebou porovnávať. V nasledujúcich
spôsoboch merania sú použité pojmy ako aktuálna hodnota $y_t$, predpovedaná
hodnota $f_t$ alebo chyba predpovede $e_t$ definovaná ako $e_t = y_t - f_t$.
Veľkosť testovacej množiny budeme označovať ako $n$~\cite{Agrawal2013}.

\subsubsection{Stredná chyba predpovede}
V literatúre označovaná ako MFE (mean forecast error). Matematickú funkciu
môžeme zapísať ako
\begin{equation}
  MFE = \frac{1}{n} \sum_{t=1}^{n} e_t
  \label{eq-mfe}
\end{equation}
Týmto spôsobom meriame priemernú odchýlku predpovedanej hodnoty od aktuálnej.
Zistíme tak smer chyby. Nevýhodou je, že kladné a záporné chyby sa vynulujú
a potom nie je možné zistiť presnú hodnotu chyby. Pri nameraní extrémnych chýb
nedochádza k žiadnej špeciálnej penalizácii. Taktiež hodnota chyby závisí od škály
meraní a môže byť ovplyvnená aj transformáciami dát. Dobré predpovede majú
hodnotu blízku 0~\cite{Agrawal2013}.

\subsubsection{Stredná absolútna chyba}
V literatúre označovaná ako MAE (mean absolute error). Patrí k jedným
z najpoužívanejších. Funkciu môžeme zapísať ako
\begin{equation}
  MAE = \frac{1}{n} \sum_{t=1}^{n} |e_t|
  \label{eq-mae}
\end{equation}
Týmto spôsobom meriame priemernú absolútnu odchýlku predpovedanej hodnoty od
aktuálnej. Zistíme tak celkový rozsah chyby, ktorá nastala počas predpovede.
Na rozdiel od merania chyby pomocou vzorca~\ref{eq-mfe} sa kladné a záporné
chyby nevynulujú, no ani napriek tomu nevieme určiť celkový smer chyby.
Na druhej strane tiež nenastáva žiadna penalizácia pri extrémnych chybách.
Hodnota chyby závisí od škály meraní a transformácií dát. Dobré predpovede majú
hodnotu čo najbližšiu 0~\cite{Agrawal2013, Gutierrez2015}.

\subsubsection{Stredná percentuálna chyba}
V literatúre označovaná ako MPE (mean percentage error). Matematicky môžeme
túto funkciu zapísať ako
\begin{equation}
  MPE = \frac{1}{n} \sum_{t=1}^{n} \frac{e_t}{y_t} \times 100
  \label{eq-mpe}
\end{equation}
Vlastnosti sú veľmi podobné ako pri MAPE v časti \ref{mape}. Chyba nám
poskytuje prehľad o priemernej chybe, ktorá sa vyskytla počas predpovede.
Naviac oproti MAPE získame prehľad o smere chyby, čo má však za následok, že
opačné znamienka sa vynulujú. O modely, ktorého chyba MPE sa blíži k 0,
nemôžeme s určitosťou tvrdiť, že funguje správne~\cite{Agrawal2013}.

\subsubsection{Stredná absolútna percentuálna chyba}
\label{mape}
V literatúre označovaná ako MAPE (mean absolute percentage error). Vzorec,
ktorým ju zapíšeme bude veľmi podobný vzorcu \ref{eq-mpe}
\begin{equation}
  MAPE = \frac{1}{n} \sum_{t=1}^{n} \Big|\frac{e_t}{y_t}\Big| \times 100
  \label{eq-mape}
\end{equation}
Pomocou tohto merania chyby získavame percentuálny prehľad o priemernej
absolútnej chybe, ktorá sa vyskytla počas predpovedi. Veľkosť chyby nezávisí od
škály merania, ale je závislá od transformácií dát. Tiež nie je možné zistiť
smer chyby a ani nenastáva žiadna penalizácia pri extrémnych
chybách~\cite{Agrawal2013}.

\subsubsection{Stredná štvorcová chyba}
V literatúre označovaná ako MSE (mean squarred error). Vzorcom ju zapíšeme ako
\begin{equation}
  MSE = \frac{1}{n} \sum_{t=1}^{n} e_t^2
  \label{eq-mse}
\end{equation}
Chyba meria priemernú štvorcovú odchýlku predpovedanej hodnoty. Opačné
znamienka sa neovplyvňujú. Neposkytuje nám pohľad na smer chyby. Zabezpečuje
penalizáciu extrémnych chýb. Zdôrazňuje fakt, že celková chyba je viac
ovplyvnená jednotlivými veľkými chybami ako viacerými malými. Nevýhodou je, že
chyba je veľmi citlivá na zmenu škály alebo transformáciu
dát~\cite{Agrawal2013}.

%-------------------------------------------------------------------------------
%   Analysis evaluation
%-------------------------------------------------------------------------------

\subsection{Zhodnotenie analýzy}
S rastúcim množstvom dát z rôznych zdrojov vzniká potreba predpovedať správanie
týchto veličín v budúcnosti. Existuje množstvo článkov, ktoré sa zaoberajú
týmto problémom. Neustále prinášajú a vylepšujú existujúce matematické modely,
ktoré sú schopné predpovedať budúce hodnoty meraných veličín. Výsledky nie sú
presné, ale chyba výpočtu je dostatočne nízka na to, aby sme takýto výsledok
mohli považovať za relevantný. Zvýšiť presnosť výsledkov je možné viacerými
metódami, či už kombináciou viacerých matematických modelov alebo ich
optimálnym nastavením. My sme sa v práci zamerali na hľadanie optimálneho
nastavenia používaných modelov.

Preskúmané predikčné modely potrebujú pre svoje správne fungovanie vstupné
parametre. Pomocou nich vieme ovplyvniť chybu predikcie. Problémom je nájdenie
takých parametrov, pre ktoré by bola chyba predikcie čo najmenšia. Hľadanie
najlepšieho riešenia by bolo časovo neprípustné, a preto budeme hľadať iba
optimálne riešenie, ktoré nám poskytne prijateľnú chybu predikcie. Je dôležité
poznamenať, že vstupné parametre, ktoré nájdu optimálnu predikciu závisia aj od
samotných dát. Ako bolo spomenuté v sekcii~\ref{time-series-analysis}, dáta
pozostávajú z viacerých zložiek a ich pomer sa môže líšiť pri rôznych
veličinách.

Na hľadanie optimálneho nastavenie predikčných algoritmov sme použili
prírodne inšpirované optimalizačné algoritmy. Ako ich názov napovedá, hľadanie
optimálneho riešenia sa vykonáva na základe správania sa nejakého živočíšneho
druhu alebo prírodného javu. Tieto algoritmy sa osvedčili ako efektívne a
rýchle riešenie problémov, ktorých prehľadávaný priestor riešení nie je možné
prehľadať celý. Algoritmy sa vyhýbajú spadnutiu do lokálnych miním a tak je
nájdené riešenie obvykle optimálne v globálnom rozsahu. A taktiež tieto algoritmy
poskytujú niekoľko konfiguračných parametrov ovplyvňujúce rýchlosť a presnosť
nájdeného riešenia. Avšak cieľom tejto práce je optimalizovať predikčné
a nie optimalizačné algoritmy a poskytnúť tak používateľovi univerzálne
a jednoduché riešenie predikčných problémov.

Dôležitou súčasťou predpovedí je vyhodnotenie ich úspešnosti porovnávaním
predpovedanej hodnoty s ich skutočnými nameranými hodnotami. Existuje množstvo
metód, ktorými vieme zmerať chybu predpovede. Niektoré nám poskytujú informáciu
o smere chyby, iné zohľadňujú extrémne chyby. Výsledky niektorých sa viažu na
škálu, v ktorej sa nachádzajú naše merania, iné sú nezávislé, merané v
percentách. Našim cieľom bolo poskytnúť používateľovi čo najviac informácií
o presnosti predpovede, a preto sú použité viaceré metódy merania chýb
predikcií.

%-------------------------------------------------------------------------------
%   Chapter 3 - Requirements specification
%-------------------------------------------------------------------------------

\newpage
\section{Špecifikácia požiadaviek}
\label{specification}
Cieľom práce je navrhnúť systém, ktorý umožní používateľovi nájsť optimálne
nastavenie vstupných parametrov predikčných metód. Používateľ bude mať
k dispozícií niekoľko predikčných a optimalizačných algoritmov, bude môcť
meniť ich parametre a výpočty vykonávať nad dátami, ktoré si sám zvolí.
K uvedeným parametrom bude poskytnuté vysvetlenie a opis zmeny správania na
základe ich zmeny. Presnosť predpovede bude určená pomocou používaných metrík
na meranie chyby predikcie.

Dáta, ktoré budeme používať počas tvorby informačného systému, pochádzajú zo
Slovenska a sú dostupné každých 15 minút. Keďže budeme vždy predpovedať práve
jednu veličinu na základe samej seba, jedno meranie by malo zodpovedať formátu
``dátum, nameraná hodnota''. Formát dátumu bude od začiatku jednotný a bude
zodpovedať norme ISO 8601. Nameraná hodnota bude reprezentovaná reálnym číslom.

Používateľ si bude môcť vyberať z ponúknutých predikčných a optimalizačných
algoritmov. K dispozícii bude mať lineárnu regresiu, stochastické modely,
regresiu založenú na podporných vektoroch, rozhodovacie stromy a prírodne
inšpirované optimalizačné algoritmy ako genetický algoritmus, umelú kolóniu
včiel alebo svorku divých vlkov. Systém bude navrhnutý tak, aby pridanie
ďalších algoritmov spôsobilo čo najmenšie zmeny v pôvodnom systéme.
Pripadá do úvahy, že samotný systém bude poskytovať rozhranie, ktoré
používateľovi umožní použiť vlastnú implementáciu iných algoritmov.

Systém bude implementovaný v jazyku R a webové používateľské rozhranie bude
vytvorené pomocou knižnice Shiny.

% %-------------------------------------------------------------------------------
% %   Chapter 4 -  Solution design
% %-------------------------------------------------------------------------------
%
% \newpage
% \section{Návrh riešenia}
% \label{solution-design}
%
% %-------------------------------------------------------------------------------
% %   Chapter 5 - Implementation
% %-------------------------------------------------------------------------------
%
% \newpage
% \section{Implementácia}
% \label{implementation}
%
% %-------------------------------------------------------------------------------
% %   Chapter 6 - Evaluation
% %-------------------------------------------------------------------------------
%
% \newpage
% \section{Zhodnotenie}
% \label{evaluation}
%
% %-------------------------------------------------------------------------------
% %   Chapter 7 - Conclusion
% %-------------------------------------------------------------------------------
%
% \newpage
% \section{Záver}
% \label{conclusion}
%
% %-------------------------------------------------------------------------------
% %   Chapter 8 - Technical documentation
% %-------------------------------------------------------------------------------
%
% \newpage
% \section{Technická dokumentácia}
% \label{documentation}

\newpage
\section{Plán do nasledujúceho semestra}

\begin{table}[H]
  \centering
  \begin{tabular}{||l|p{0.65\textwidth}||}
    \hline \hline
    \multicolumn{1}{|c|}{\textbf{\begin{tabular}[c]{@{}c@{}}Poradie týždňa \\ v letnom semestri\end{tabular}}} & \multicolumn{1}{c|}{\textbf{Popis plánovanej činnosti}}   \\ \hline
    \hline
    1. týždeň    &  úprava vstupných dát na požadovaný formát, aplikovanie predikčných algoritmov             \\ \hline
    2. týždeň    &  aplikovanie predikčných a optimalizačných algoritmov                                      \\ \hline
    3. týždeň    &  implementácia optimalizačných algoritmov, ktorým chýba podpora knižníc v jazyku R         \\ \hline
    4. týždeň    &  implementácia optimalizačných algoritmov, tvorba kostry grafického rozhrania aplikácie    \\ \hline
    5. týždeň    &  tvorba jednotného rozhrania medzi aplikáciou a grafickým rozhraním                        \\ \hline
    6. týždeň    &  implementácia grafického rozhrania a následne prepojenie s aplikáciou                     \\ \hline
    7. týždeň    &  ošetrenie neplatných akcií vykonané používateľom                                          \\ \hline
    8. týždeň    &  testovanie aplikácie na rôznych vstupných dátach používateľmi                             \\ \hline
    9. týždeň    &  tvorba technickej dokumentácie aplikácie                                                  \\ \hline
    10. týždeň   &  testovanie algoritmov, dokumentácia a porovnanie algoritmov                               \\ \hline
    11. týždeň   &  tvorba prezentácie a príprava na obhajobu projektu                                        \\ \hline
    \hline
  \end{tabular}
\end{table}

%-------------------------------------------------------------------------------
%   Bibliography
%-------------------------------------------------------------------------------

\newpage
\addcontentsline{toc}{section}{Literatúra}
\bibliographystyle{iso-690/slovakiso}
\bibliography{bibliography}

\end{document}

% \begin{table}[H]
%   \caption{Čo ešte určite musím stihnúť VYMAZAŤ (podľa dôležitosti)}
%   \centering
%   \begin{tabular}{|p{0.9\textwidth}|}
%     \hline
%     % pridať ACO od Marca Doriga      \\ \hline
%     % pridať regresné stromy          \\ \hline
%     % nájsť a pridať vzorec pre SVR   \\ \hline
%     pridať zhodnotenie analýzy      \\ \hline
%     pridať špecifikáciu požiadaviek (opis use case)   \\ \hline
%     % pridať exponencionálne vyrovnávanie               \\ \hline
%     % pridať učenie súborov klasifikátorov              \\ \hline
%   \end{tabular}
% \end{table}

% \paragraph{Autoregressive model}
% môže modelovať profil záťaže za predpokladu, že zátaž je lineárnou kombináciou
% predchádzajúcich záťaží\cite{KumarSingh2013}.

% \paragraph{Support Vector Machine based Techniques}
% je metóda analyzujúca dáta a rozpoznávajúca vzory, používaná na roztriedenie
% a regresnú analýzu, kombinuje zovšeobecnené riadenie
% s technikou ??????\cite{KumarSingh2013}.
%
% \paragraph{Support Vector Machine}
% je ML algoritmus používaný ako na klasifikáciu tak na regresiu
% support vector sú koordináty jednotlivých meraní napr. muž a žena a ich merané veličny reprezentované na osy, ktoré sú hraničnými elementami rôznych skupín
% maximalizuje rozmädzie medzi support vektormi jednej kategórie a support vektormi druhej kategórie, rozhodovacia funkcia je definovaná podmnožinou testovacej vzorky (jednotlivé supprot vektory)
% v 2D sú kategórie oddelené čiarou vo viacrozmerných dimenziách rovinou
%
% \paragraph{Incremental SVM}
% základom je pridávanie % http://www.jmlr.org/papers/volume7/laskov06a/laskov06a.pdf
% nový bod má najskôr pridelenú váhu 0, ak toto pridelenie nie je optimálnym riešením, teda bod sa môže stať support vectorom,
% váhy ostatných vektorov a rozhodovací prah musia byť aktualizované kvôli získaniu optimálneho riešenia nad novou množinou support vektorov
%
% \paragraph{Linear SVM}
% linárna kombinácia elementov (features, črty) značí, že sa jedná aj o lineárny klasifikátor  % http://stackoverflow.com/questions/6160495/support-vector-machines-a-simple-explanation
% napr ak (w1 * x1 + w2 * x2) > C potom element patrí do skupiny A, hodnotami x1 a x2 je element definovaný, tak ako je bod definovaný x a y súradnicou
% w je váha a C rozhodovacií prah, čiže ak nejaký ohodnotený element neprekročí hranicu spadá do jednej skupiny, ak prekročí spadá do druhej
%
% \paragraph{Concept drift}
% je správanie premennej, ktorú sa snažím predikovať sa môže časom meniť,
% čím sa postupne stáva model menej a menej presný\cite{Grmanova2016}.
%
% \paragraph{Online algorithm}
% spracováva vstup sériovo kúsok po kúsku, vstupné dáta nie sú dostupné na začiatku výpočtu % http://stackoverflow.com/questions/11496013/what-is-the-difference-between-an-on-line-and-off-line-algorithm
% musí spracovať vstup v jednej iterácií bez žiadnej podrobnej znalosti budúcich vstupov % https://xlinux.nist.gov/dads/HTML/online.html
% viac dát, časové obmedzenia, môže sa časom meniť % http://stats.stackexchange.com/questions/897/online-vs-offline-learning
%
% \paragraph{Offline algorithm}
% rieši problém od začiatku so všetkými vstupnými dátami % http://stackoverflow.com/questions/11496013/what-is-the-difference-between-an-on-line-and-off-line-algorithm
% vopred je daná celá séria vstupov % https://xlinux.nist.gov/dads/HTML/offline.html

% \paragraph{Kernel trick}
% problém nie je lineárne separovateľný, originálny nelineárny priestor % http://stats.stackexchange.com/questions/3947/help-me-understand-support-vector-machines
% je premietnutý do viacrozmerného priesotru pomocou nejakej nelineárnej transofrmácia s očakávaním, že to problém už bude riešiteľný
%
% \paragraph{Extreme learning machine}
% je novovznikajúca technika učenia poskytujúca efektívne % http://cherup.yonsei.ac.kr/files/Paper/2013_IEEE%20Intelligent%20Systems%20-%20Off%20line%20version_A%20System%20for%20Signature%20Verification%20Based%20on%20Horizontal%20and%20Vertical%20Components%20in%20Hand%20Gestures.pdf
% a zjednotené riešenie na všeobecné dopredné siete ako
% neurónové siete, RBF siete alebo kernelové učenie

% Časový rád je súbor meraní presne definovaných veličín získavaných opakovanými
% meraniami. Dáta zbierané zriedkavo alebo jednorázovo nepovažujeme za časový rád.
% Pozorované časové rády možno rozložiť na 3 zložky a to trendovú, sezónnu
% a nepravidelnú\cite{AustralianBureau}.

% Trendová zložka predstavuje smer veličiny v dlhodobom horizonte a máva klesajúci
% alebo stúpajúci charakter. Na druhej strane, sezónna zložka má cyklický
% charakter a dĺžka cyklu sa viaže napr. ku dňu, týždnu či roku. Nepravidelná
% zložka reprezentuje náhodné zmeny v prostredí, ktoré nie sú relevantné pre
% predpoveď časových rádov. Pri trénovaní modelu sa ich snažíme odfiltrovať
% optimálnou mierou natrénovania modelu.

% Pri predpovedaní časových radov ako napr. meraní odberu elektriky vznikajú 2 typy tvz. Concept drift.
% \textbf{Concept drift} je zmena správania veličiny, ktorú sa snažíme
% predpovedať. Model sa tak stáva postupne nepresný a je potrebné aby sa tejto
% zmene prispôsobil. Prvým typom je trvalá alebo dočasná zmena spôsobená
% ekonomickými alebo ekologickými faktormi. Druhým typom je sezónna zmena,
% spôsobená zmenami ročných období a množstvom denného svetla. Sezónnu zmenu
% môžeme pozorovať na úrovni dní, týždňov alebo rokov. Kvôli tomu je nutné
% v každom modely rozdeľovať tieto 2 typy concept drift\cite{Grmanova2016}.

% \subsection{Reziduálna zložka}
% Ostáva v časovom rade po odstránení trendovej, cyklickej a sezónnej zložky.
% Je tvorená náhodnými pohybmi v priebehu časového radu. Tiež pokrýva chyby
% v meraní. Obvykle sa predpokladá, že reziduálna zložka je biely šum, teda
% nekorelované náhodné veličny s nulovou strednou hodnotou\cite{http://www.math.sk/mpm/otazka_30.pdf}.

% ε-insensitive loss function defined
% \[
%     L_{\varepsilon}(y, f(x, w)) =
%     \begin{cases}
%       0 \text{ ak } |y - f(x, w)| \leq \varepsilon \\
%       |y - f(x, w)| - \varepsilon \text{ inak } \\
%     \end{cases}
% \]

% \begin{equation}
%   fitness = \delta + P
%   \label{eq-fitness}
% \end{equation}
% Faktor $\delta$ predstavuje rozdieľnosť výrazu aktuálneho chromozónu od
% cieľového chromozónu \cite{Simonova2007}

% \paragraph{Logistický regresný model}
% Nelineárna diskriminantná štatistická metóda. V \textbf{binary response} modely
% os $y$ zvyčajne reprezentuje individuálnu alebo experimentálnu jednotku. $Y$ môže
% nadobúdať hodnoty 0 alebo 1 pre situácie kedy udalosť nastane alebo nenastane.
% Os $x$ reprezentuje vysvetľujúcu veličinu ako vektor, ktorý môže znázorňovať
% pravdepodobnosť udalosti $(Y = 1)$~\cite{Li2010}.

%-------------------------------------------------------------------------------
%   Naive methods
%-------------------------------------------------------------------------------

% \subsubsection{Naivné metódy}
% Predpovede sú vytvárané pomocou posledných hodnôt alebo ich priemerov.
%
% \paragraph{Seasonal naïve method}
% Poslednú nameranú hodnotu použijeme ako predpoveď pre nasledujúce obdobie. Ak
% sú naše dáta vysoko závisle od ročného obdobia, je lepšie použiť na predpoveď
% hodnotu z rovnakého obdobia, napr. z minulého roka~\cite{Grmanova2016}.
%
% \paragraph{Naïve average long-term method}
% Predpokladá, že dáta obsahujú vzory, ktoré nie sú závislé od ročných období.
% Kvôli tomu sú časové rady lokálne stabilné s pomaly meniacim sa priemerom.
% Hodnotu, ktorú použijeme ako predpoveď je iba priemorom viacerých posledných
% hodnôt~\cite{Grmanova2016}.
%
% \paragraph{Naïve In median long-term method}
% Táto metóda je alternativou k predchádzajúcej metóde. Keďže priemerom nedokáže
% model dostatočne rýchlo reagovať na rapídne výkyvy a abnormality, lepšie
% výsledky dosiahneme nahradením priemeru za median posledných \textit{n}
% meraní~\cite{Grmanova2016}.

% Predpovedanie veličín, ktoré My sme sa zamerali na predpovedanie veličín , ktorých predpoveď môže závisieť od
% iných veličín, ale
%
% V práci sme sa zamerali na p Predpovedanie sa tak stáva  čo či tý Väčšiu spotrebu elektrickej energie máme počas práce či
%
% zábave, naopak len minimum spotrebujeme počas spánku. Už len z tohto príkladu
% je zrejmé, spotreba elektriny ľudí sa líši od ich správania, zvyklostí či
% možností, ale aj faktorov, ktoré pôsobia na nás všetkých ako je počasie, deň
% v týždni, pracovné voľná či kultúrne tradície. Faktom je, že spotreba elektriky
% bude naďalej rásť, keďže celý svet postupne minimalizuje závislosť na
% neobnoviteľných zdrojoch energie a náhradou má byť práve elektrická energia.
%
% Pre efektívnu výrobu a distribúciu elektriky je potrebné vedieť predpovedať
% jej spotrebu, čím zamedzíme ekonomickým stratám ako na strane výrobcu tak aj na
% strane spotrebiteľa. Predpoveď by sa, v ideálnom prípade, mala líšiť od
% skutočnej hodnoty minimálne a

% V literatúre označovaný ako ABC algoritmus (Artifical bee colony) je pomerne
% nový medzi rojovými algoritmami. Princíp je založený na biologickom procese,
% správaní medonosných včiel pri hľadaní potravy. Hlavný mechanizmus ktorým včely
% optimalizujú množstvo procesov je tzv. včelí tanec (\textbf{waggle dance}),
% ktorým včely lokalizujú zdroje potravy a nachádzajú ďalšie~\cite{Chavan2015}.
%
% Každá včela na pracujúca v roji sa spolupodiela na tvorbe celého systému na
% globálnej úrovni. Správanie systému je určené lokálnym správaním, kde spolupráca
% a zladenie jedincov vedie k štruktúrovanému kolaboračnému systému~\cite{Chavan2015}.
%
% Algoritmus funguje na princípe, že včely nájdu najviac výnosný zdroj
% s použitím, čo najmenšej energie. \textbf{Foragers} (zrejme robotnice hľadajúce zdroj jedla) uvažujú
% presúvanie sa medzi zdrojmi nektárov na základe kvality alebo zisku zdroju.
% Algoritmus poskytuje samo-manažovateľné a samo-organizované riešenie, vo svojej
% podstate decentralizované, pre daný problém~\cite{Buhussain2016}.

% %-------------------------------------------------------------------------------
% %   Ensemble learning
% %-------------------------------------------------------------------------------
%
% \subsubsection{Učenie súborov klasifikátorov}
% Používa sa na jednodňovú predikciu. Ak \textit{h} je počet meraní, ktoré sú
% denne dostupné, v deň \textit{t} sa vykoná \textit{h} predikcií podľa váženého
% priemeru \textit{m} modelmi. Nasledujúci deň sa vypočíta chyba predpovede,
% na základe ktorej sa znova prepočítajú váhy a každý model sa
% aktualizuje\cite{Grmanova2016}.
%
% Učenie súborov klasifikátorov môžeme rozdeliť na homogénne a heterogénne učenie.
%
% \paragraph{Homogénne učenie súborov klasifikátorov}
% Pozostáva z modelov rovnakého typu, ktoré sa učia na rôznych podmnožinách
% datasetu.
% \paragraph{Heterogénne učenie súborov klasifikátorov}
% Aplikuje rôzne typy modelov nad rovnakými dátovými množinami\cite{Grmanova2016}.
%
% %-------------------------------------------------------------------------------
% %   Exponential smoothing
% %-------------------------------------------------------------------------------
%
% \subsubsection{Exponencionálne vyrovnávanie}
%
% Táto metóda, tak ako vyššie popísané metódy, najprv na základe dát
% z predchádzajúcich meraní vytvorí model. Ten sa v ďalej použíje na
% redpovedanie budúcich dát. Tento vzťah môžeme matematicky zapísať ako
% \begin{equation}
%   y(t) = \beta(t)^T f(t) + \varepsilon(t)
%   \label{eq-expo}
% \end{equation}
% Vo vzorci \ref{eq-expo} sa opäť nachádza biely šum $\varepsilon(t)$. Hodnota
% $\beta(t)$ predstavuje \textbf{coefficient vector} a $f(t)$
% \textbf{fitting function}~\cite{Mahalakshmi2016}.
%
% Jednou z predikčných metód časových radov, ktorá používa exponencionálne
% vyrovnávanie je aj Wintersova metóda. Pri sezónnych dátach sú použité 3
% vyrovnávacie parametre a to trend, sezónosť a stacionárnosť. Analýza
% a predpoveď priamym aplikovaním Wintersovej metódy môže byť náročné, keďže
% dátové množiny väčšinou obsahujú \textbf{riedke pozorovania (sparse values)}.
% Tento problém môžeme vyriešiť kombináciou ostatných metód, ako je napr.
% autoregresívny model či spektrálna analýza, s exponencionálnym
% vyrovnávaním~\cite{Mahalakshmi2016}.

% %-------------------------------------------------------------------------------
% %   Ant colony optimization
% %-------------------------------------------------------------------------------
%
% \subsubsection{Kolónia mravcov}
% Pri tomto algoritme, mravce tiež opúšťajú mravenisko, kvôli hľadaniu zdrojov
% potravy náhodne. Potom vyhodnotia kvalitu zdroja potravy a donesú ho naspäť do
% mraveniska. Zanechávajú pri tom na zemi chemické stopy. Sila týchto stôp závisí
% od kvality nájdeného zdroja potravy. Mnoho výskumov využíva tento algoritmus na
% riešenie NP problémov, ako napríklad problém obchodných cestujúcich,
% vyfarbovanie grafov, smerovnaie áut alebo plánovacie problémy. Používa sa aj pri
% \textbf{cloud computing} na nájdenie optimálneho riešenia pri plánovaní úloh
% pre virtuálne servery~\cite{Buhussain2016}.
%
% Keď mravce hľadajú potravu prvý krát, hľadajú náhodne až kým nenájdu zdroj
% potravy. Zanechávajú pri tom za sebou chemickú stopu nazývanú feromón, ktorá
% tak vedie k zdroju. Tá následne priťahuje ostatné mravce k tomuto zdroju
% potravy. Tento proces pokračuje pokiaľ mravce nenájdu najkratšiu cestu vedúcu
% ku konkrétnemu zdroju potravy. Najkratšia cesta je určená naakumulovaným
% množstvom feromónov na ceste k zdroju potravy~\cite{Buhussain2016}.

%-------------------------------------------------------------------------------
%   Chapter X - Conclusion
%-------------------------------------------------------------------------------

% \chapter{Záver}
% Tu bude záver

% Kapitola \chapter{Nazov}
% Necislovana kapitola \chapter*{Nazov}% underline \underline{science}
% Pokapitola (Section) \section{Nazov}
% Subsection \subsection{Nazov}
% Paragraph \paragraph{Nazov}
% Ak niečo nechceme číslovať, použijeme *, avšak, ak to chceme v obsahu, musíme to do neho pridať

% \Huge, \huge, \LARGE, \Large, \large, \normalsize, \small, \footnotesize, \tiny
% italic \textit{accident}.
% bold \textbf{greatest}
% -1 part     1 section     3 subsubsection  5 subparagraph
%  0 chapter  2 subsection  4 paragraph
