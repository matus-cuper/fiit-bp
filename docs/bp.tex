% Bakalársky projekt 2016/2017
% Matúš Cuper

%-------------------------------------------------------------------------------
%   PACKAGES AND DOCUMENT CONFIGURATION
%-------------------------------------------------------------------------------

\documentclass[12pt,oneside,slovak,a4paper]{book}

% \usepackage{float}
\usepackage[slovak]{babel}
% \usepackage[T1]{fontenc}
\usepackage[utf8]{inputenc}
% \usepackage{graphicx}
% \usepackage{url}
\usepackage{titlesec}																														% remove "Chapter N" header from chapter
\usepackage{times}																															% Times New Roman
% \usepackage{hyperref}
% \usepackage{cite}
% \usepackage{times}
% \usepackage[dvips,dvipdfm,a4paper,centering,textwidth=14cm,top=4.6cm,headsep=.6cm,footnotesep=1cm,footskip=0.6cm,bottom=3.8cm]{geometry}
\usepackage[a4paper, centering,
											left=30mm, top=20mm, right=20mm, bottom=20mm]{geometry}		% set page margins


% \pagestyle{headings}
\titleformat{\chapter}[display] {\normalfont\bfseries}{}{0pt}{\Large}						% remove "Chapter N" header from chapter

%-------------------------------------------------------------------------------
%   TITLE PAGES
%-------------------------------------------------------------------------------

\begin{document}
\begin{titlepage}
	\centering
	{\large \textbf{SLOVENSKÁ TECHNICKÁ UNIVERZITA V BRATISLAVE} \par}
	\vspace{0.5cm}
	{\large \textbf{Fakulta informatiky a informačných technológií} \par}
	\vspace{6cm}
	{\huge\bfseries Optimalizácia konfiguračných parametrov predikčných metód \par}
	\vspace{2cm}
	{\scshape\large \textbf{Bakalárska práca} \par}
	\vspace{12.5cm}
	\noindent{\bfseries\large 2016 \hfill Matúš Cuper }
	\vfill
\end{titlepage}

\begin{titlepage}
	\centering
	{\large \textbf{SLOVENSKÁ TECHNICKÁ UNIVERZITA V BRATISLAVE} \par}
	\vspace{0.5cm}
	{\large \textbf{Fakulta informatiky a informačných technológií} \par}
	\vspace{6cm}
	{\huge\bfseries Optimalizácia konfiguračných parametrov predikčných metód \par}
	\vspace{2cm}
	{\scshape\large \textbf{Bakalárska práca} \par}
	\vspace{5.5cm}
	\raggedright
	{\normalsize Študijný program: \par}
	{\normalsize Číslo študijného odboru: \par}
	{\normalsize Názov študijného odboru: \par}
	{\normalsize Školiace pracovisko: \par}
	{\normalsize Vedúci záverečnej práce: \par}
	\vspace{4.5cm}
	\noindent{\bfseries\large Bratislava 2016 \hfill Matúš Cuper }
\end{titlepage}

% \podakovanie
% \abstrakt
% \abstract
% \prehlasenie

\tableofcontents

%-------------------------------------------------------------------------------
%   Chapter 1 - Introduction
%-------------------------------------------------------------------------------

\chapter{Úvod}
Tu bude úvod

%-------------------------------------------------------------------------------
%   Chapter 2 - Analysis of predictive algorithms
%-------------------------------------------------------------------------------

\chapter{Delenie predičkných algoritmov}

\section{Naivné prístupy}
\subsection{}


\section{Klasické prístupy}

\subsection{Stupňovitý regresný model}
\subsection{Neurónové siete}
\subsection{Rozhodovací strom}
\subsection{AR}
\subsection{MA}
\subsection{ARMA}
\subsection{ARIMA}
\subsection{Box-Jekins metodika}

\section{Prírastkové učenie}
\subsection{Prírastkové SVM}
\subsection{Extrémne strojové učenie}
\subsection{Prírastková ARIMA}

\section{Učenie súborov klasifikátorov}
Dva prístupy: homogénne a heterogénne učenie ale aj ich kombinácia
\textbf{Homogénne učenie súborov klasifikátorov} pozostáva z modelov rovnakého
typu, ktoré sa učia na rôznych podmnožinách datasetu.
\textbf{Heterogénne učenie súborov klasifikátorov} aplikuje rôzne typy modelov
nad rovnakými dátovými množinami\cite{Grmanova2016}.
\subsection{Bagging}
\subsection{Boosting}
\subsection{Adaboost}
\subsection{Stacked generalization}
\subsection{Mixture of experts}

\paragraph{Regression model}
je funkcia (f), ktorou sa snažíme odhadnúť závislú premmenú (Y), pomocou
nezávislej premennej (X) a neznámych parametrov (Beta), (Beta) odhadujeme na
základe dát, (f) je známa na základe vzťahov medzi (Y) a (X)\cite{Grmanova2016}.
najpoužívanejšia štatistická metóda, modeluje vzťah závislej premennej
(spotreba elektriky) a nezávislej premmennej, čiže ostatných faktorov
(počasie, deň v týždni, odoberateľ), spotreba tak môže byť rozdelená na
štandardný smer (trend) a trend lineárne závislý od niektorých faktorov
vplývajúcich na spotrebu\cite{KumarSingh2013}.

\paragraph{Multiple regression model}
najpopulárnejšia metóda, často používa na predpoveď spotreby ovplyvnenej
množstvom faktorov (meteologické, cena elektriky, ekonomický
nárast)\cite{KumarSingh2013}.

\paragraph{Autoregressive model}
môže modelovať profil záťaže za predpokladu, že zátaž je lineárnou kombináciou
predchádzajúcich záťaží\cite{KumarSingh2013}.

\paragraph{Autoregressive Moving-Average model}
model reprezentuje súčastnú hodnotu časového rádu linárne na základe jeho hodnôt
a hodnôt bieleho šumu v predchádzajúcich periódach\cite{KumarSingh2013}.

\paragraph{Ensemble model}
používa sa na jednodňovú predikciu, h je počet meraní, ktoré sú denne dostupné,
v deň t sa vykoná h predikcií podľa váženého priemeru m modelmi, nasledujúci
deň sa vypočíta chyba predpovede, na základe ktorej sa znova prepočítajú váhy
a každý model sa aktualizuje\cite{Grmanova2016}.

\paragraph{Seasonal naïve method}
poslednú zmeranú hodnotu použijem ako predpoveď alebo pre high seasonal data
použijem už nameranú hodnotu z rovnakého obdobia (napr. rok
dozadu)\cite{Grmanova2016}.

\paragraph{Naïve average long-term method}
je založené na predpoklade non-seasonal patterns, predpokladá, že časové rády sú
lokálne stabilné s pomaly meniacim sa priemerom, predpovedaná hodnota je
priemorom viacerých hodnôt\cite{Grmanova2016}.

\paragraph{Naïve In median long-term method}
je alternatíva k predchádzajúcej metóde, priemer nie je schopný reagovať na
rapídne výkyvy a abnormality, lepšia možnosť je preto spraviť median
z posledných n časových radov\cite{Grmanova2016}.

\paragraph{Stochastic Time Series}
metódy časových radov sú založené na predpoklade, že dáta majú vnútornú
štruktúru, ako napr. autokoreláciu, trend či sezónnu variáciu, najprv sa
precízne zostaví vzor zodpovedajúci dostupným dátam a potom sa predpovie
hodnota\cite{KumarSingh2013}.

\paragraph{Support Vector Machine based Techniques}
je metóda analyzujúca dáta a rozpoznávajúca vzory, používaná na roztriedenie
a regresnú analýzu, kombinuje zovšeobecnené riadenie
s technikou ??????\cite{KumarSingh2013}.


\chapter{Záver} \label{zaver}
Tu bude záver

%-------------------------------------------------------------------------------
%   Bibliography
%-------------------------------------------------------------------------------

\bibliography{bibliography}
\bibliographystyle{ieeetr}

\end{document}



% Kapitola \chapter{Nazov}
% Necislovana kapitola \chapter*{Nazov}% underline \underline{science}
% Pokapitola (Section) \section{Nazov}
% Subsection \subsection{Nazov}
% Paragraph \paragraph{Nazov}
% Ak niečo nechceme číslovať, použijeme *, avšak, ak to chceme v obsahu, musíme to do neho pridať

% \Huge, \huge, \LARGE, \Large, \large, \normalsize, \small, \footnotesize, \tiny
% italic \textit{accident}.
% bold \textbf{greatest}
