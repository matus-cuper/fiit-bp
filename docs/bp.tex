% Bakalársky projekt 2016/2017
% Matúš Cuper

%-------------------------------------------------------------------------------
%   PACKAGES AND DOCUMENT CONFIGURATION
%-------------------------------------------------------------------------------

\documentclass[a4paper,slovak,12pt,appendix]{article}

% \usepackage{float}
\usepackage[slovak]{babel}
% \usepackage[T1]{fontenc}
\usepackage[utf8]{inputenc}
% \usepackage{graphicx}
% \usepackage{url}
\usepackage{amssymb}                                                            % for real numbers sign
\usepackage{amsmath}                                                            % for fractions
\usepackage{indentfirst}                                                        % indent first line after section
% \usepackage{titlesec}																													% remove "Chapter N" header from chapter
\usepackage{times}																															% Times New Roman
\usepackage[unicode]{hyperref}																									% enable hyper references in table of content
\hypersetup{																																		% with default colours for links
    colorlinks,
		pageanchor=false,
    citecolor=black,
    filecolor=black,
    linkcolor=black,
    urlcolor=black
}

% \usepackage{cite}
% \usepackage{times}
% \usepackage[dvips,dvipdfm,a4paper,centering,textwidth=14cm,top=4.6cm,headsep=.6cm,footnotesep=1cm,footskip=0.6cm,bottom=3.8cm]{geometry}
\usepackage[a4paper, centering,
											left=30mm, top=20mm, right=20mm, bottom=20mm]{geometry}		% set page margins


% \pagestyle{headings}
%\titleformat{\chapter}[display] {\normalfont\bfseries}{}{0pt}{\Large}						% remove "Chapter N" header from chapter

%-------------------------------------------------------------------------------
%   TITLE PAGES
%-------------------------------------------------------------------------------

\begin{document}
\begin{titlepage}
	\centering
	{\large \textbf{SLOVENSKÁ TECHNICKÁ UNIVERZITA V BRATISLAVE} \par}
	\vspace{0.5cm}
	{\large \textbf{Fakulta informatiky a informačných technológií} \par}
	\vspace{6cm}
	{\huge\bfseries Optimalizácia konfiguračných parametrov predikčných metód \par}
	\vspace{2cm}
	{\scshape\large \textbf{Bakalárska práca} \par}
	\vspace{12.5cm}
	\noindent{\bfseries\large 2016 \hfill Matúš Cuper }
	\vfill
\end{titlepage}

\begin{titlepage}
	\centering
	{\large \textbf{SLOVENSKÁ TECHNICKÁ UNIVERZITA V BRATISLAVE} \par}
	\vspace{0.5cm}
	{\large \textbf{Fakulta informatiky a informačných technológií} \par}
	\vspace{6cm}
	{\huge\bfseries Optimalizácia konfiguračných parametrov predikčných metód \par}
	\vspace{2cm}
	{\scshape\large \textbf{Bakalárska práca} \par}
	\vspace{5.5cm}
	\raggedright
	{\normalsize Študijný program: Informatika \par}
	{\normalsize Číslo študijného odboru: 9.2.1 \par}
	{\normalsize Názov študijného odboru: Informatika \par}
	{\normalsize Školiace pracovisko: Ústav informatiky a softvérového inžinierstva, FIIT STU Bratislava \par}
	{\normalsize Vedúci záverečnej práce: Ing. Marek Lóderer \par}
	\vspace{4.5cm}
	\noindent{\bfseries\large Bratislava 2016 \hfill Matúš Cuper }
\end{titlepage}

%-------------------------------------------------------------------------------
%   ANOTATION
%-------------------------------------------------------------------------------

\newpage
\thispagestyle{plain}
\vspace*{1.5cm}
\begin{center}
  \begin{Large}
    \textbf{Anotácia} \par
  \end{Large}
\end{center}
{Slovenská technická univerzita v Bratislave \par}
{FAKULTA INFORMATIKY A INFORMAČNÝCH TECHNOLÓGIÍ \par}
{Študijný program: Informatika \par}
{Autor: Matúš Cuper \par}
{Bakalárska práca: Optimalizácia konfiguračných parametrov predikčných metód \par}
{Vedúci práce: Ing. Marek Lóderer \par}
{máj 2016 \\} \\
Tu bude text slovenskej anotácie

\newpage
\thispagestyle{plain}
\vspace*{1.5cm}
\begin{center}
  \begin{Large}
    \textbf{Annotation} \par
  \end{Large}
\end{center}
{Slovak University of Technology Bratislava \par}
{FACULTY OF INFORMATICS AND INFORMATION TECHNOLOGIES \par}
{Degree Course: Computer Science \par}
{Author: Matúš Cuper \par}
{Bachelor thesis: Optimizing configuration parameters of prediction methods \par}
{Supervisor: Ing. Marek Lóderer \par}
{May 2016 \\} \\
Tu bude text anglickej anotácie

%-------------------------------------------------------------------------------
%   Declaration
%-------------------------------------------------------------------------------

\newpage
\thispagestyle{plain}
\vspace*{15cm}
\begin{large}
  \noindent \textbf{POĎAKOVANIE} \\
\end{large}
\noindent
Tu bude poďakovanie

\newpage
\thispagestyle{plain}
\vspace*{15cm}
\begin{large}
  \noindent \textbf{ČESTNÉ PREHLÁSENIE} \\
\end{large}
\noindent
Tu bude prehlásenie \\
\vspace*{0.5cm}\\
\hspace*{10cm}............................\\
\hspace*{10.7cm} Matúš Cuper

%-------------------------------------------------------------------------------
%   Table of contents
%-------------------------------------------------------------------------------

\newpage
\tableofcontents

%-------------------------------------------------------------------------------
%   Chapter 1 - Problem analysis
%-------------------------------------------------------------------------------

\newpage
\section{Analýza problému}

%-------------------------------------------------------------------------------
%   Time series
%-------------------------------------------------------------------------------

\subsection{Časové rady}
Časový rad je množina dátových bodov nameraná v čase postupne za sebou.
Matematicky je definovaný ako množina vektorov x(t), kde t reprezentuje
uplynulý čas. Premenná x(t) je považovaná za náhodnú premennú.
Merania v časových radoch sú usporiadané v chronologicky
poradí \cite{Agrawal2013}.

Časové rady delíme na spojité a diskrétne. Pozorovania pri spojitých časových
radoch sú merané v každej jednotke času, zatiaľ čo diskrétne obsahujú iba
pozorovania v diskrétnych časových bodoch. Hodnoty toku rieky, teploty
či koncentrácie látok pri chemickom procese môžu byť zaznamenané ako spojitý
časový rad. Naopak, populácia mesta, produkcia spoločnosti alebo kurzy mien
reprezentujú diskrétny časový rad. Vtedy sú pozorovania oddelené rovnakými
časovými intervalmi, napr. rokom, mesiacom či dňom \cite{Agrawal2013}. V našom
prípade sú namerané dáta dostupné každú celú štvrťhodinu.

Cieľom predikcií časových radov je predpovedať hodnotu premennej v budúcnosti
na základe doteraz nameraných dátových vzoriek. Preto je potrebné nájsť funkciu,
ktorá predpovedá hodnotu časového radu v budúcnosti konzistentne
a objektvíne \cite{Sapankevych2009}.

Časové rady sú najčastejšie vizualizované ako graf, kde pozorovania sú na
osy y a plynúci čas na osy x.

\subsubsection{Zložky časových radov}
Pri predpovedaní časových radov ako napr. meraní odberu elektriky vznikajú
2 typy trendov. Prvým typom je trvalá alebo dočasná zmena spôsobená
ekonomickými alebo ekologickými faktormi. Druhým typom je sezónna zmena,
spôsobená zmenami ročných období a množstvom denného svetla. Môžeme ju pozorovať
na úrovni dňov, týždňov alebo rokov. Veličina, ktorú sa snažíme predpovedať
postupne mení svoje správanie a model sa tak stáva nepresným. Kvôli tomu je
nutné v každom modely rozdeľovať tieto typy tendencií, aby sme vedeli model
zmenám prispôsobiť \cite{Grmanova2016}.

Vo všeobecnosti sú časové rady zložené zo 4 hlavných zložiek, ktoré môžeme
odlíšiť od pozorovaných dát. Jedná sa o trendovú, cyklickú, sezónnu
a reziduálnu zložku \cite{Agrawal2013}.

\paragraph{Trendová zložka}
V dlhodobom časovom horizonte majú časové rady tendenciu klesať, rásť alebo
stagnovať. Príkladom môže byť nárast populácie či klesajúca
úmrtnosť \cite{Agrawal2013}.

\paragraph{Cyklická zložka}
V strednodobom časovom horizonte sa vyskytujú okolnosti, ktoré spôsobujú
cyklické zmeny v časových radoch. Dĺžka periódy je 2 a viac rokov. Táto zložka
je zastúpená najmä pri ekonomických časových radoch napríklad podnikateľský
cyklus pozostávajúci zo 4 fáz, ktoré sa stále opakujú \cite{Agrawal2013}.

\paragraph{Sezónna zložka}
Ide o kolísanie počas ročných období. Dôležitými faktormi pri tom sú napr.
klimatické podmienky, tradiície alebo počasie. Napríklad predaj zmrzliny sa
v lete zvyšuje, ale počet predaných lyžiarskych súprav klesá \cite{Agrawal2013}.

\paragraph{Reziduálna zložka}
Jedná sa o veličinu, ktorá nemá žiadny opakovateľný vzor a ani dlhodobý trend.
V časových radoch má nepredvídateľný vplyv na pozorovanú veličinu. V štatistike
zatiaľ nie je definovaná metóda na jej meranie. Označuje sa aj ako náhodná
zložka alebo biely šum. Je spôspobená nepredvídateľnými a nepravideľnými
udalosťami \cite{Agrawal2013}.

%-------------------------------------------------------------------------------
%   Analysis of prediction algorithms
%-------------------------------------------------------------------------------

\subsection{Analýza predičkných algoritmov}
Na základe množstva predikčných

%-------------------------------------------------------------------------------
%   Linear regression
%-------------------------------------------------------------------------------

\subsubsection{Lineárna regresia}
Najpoužívanejšia štatistická metóda, ktorá modeluje vzťah závislej premennej
a vysvetľujúcej premmennej. Závislú premmenu predstavuje veličina, ktorú sa
snažíme predpoved, čo je v našom prípade spotreba elektriky. Vysvetľujúca
premenná v sebe zahŕňa rôzne faktory, ktoré ovplyvňujú závislú premennú.
Môžeme si pod tým predstaviť deň v týždni, počasie, tradície alebo rôzne
udalosti, ktoré majú vplyv na predpoveď. \cite{KumarSingh2013}.

Predpokladajme typický regresný problém. Dáta pozostávajúce z množiny \textit{n}
meraní majú formát $\left\{(x_1, \textit{f}(x_1)), ..., (x_n, \textit{f}(x_n))\right\}$.
Úlohou regresie je odvodiť funkciu \textit{\^{f}} z dát, kde
\begin{equation}
  \hat{f} : X \to \mathbb{R} \text{, kde, } \hat{f}(x) = f(x), \forall x \in \textit{X},
  \label{eq-regresia}
\end{equation}
Funkcia \textit{f} vo vzorci \ref{eq-regresia} reprezentuje reálnu neznámu
funkciu. Algoritmus použitý na odvodenie funkcie \textit{\^{f}} sa nazýva
indukčný algoritmus alebo žiak. Funkcia \textit{\^{f}} sa nazýva model alebo
prediktor. Obvykle je úlohou regresie minimalizovať odchýlku funkcie pre
štvorcovú chybu, konkrétne strednú štvorcovú chybu MSE \cite{Mendes-Moreira2012}.

\paragraph{Lineárna regresná analýza}
Regresná analýza je štatistická metóda používaná na modelovanie vzťahov, ktoré
môžu existovať medzi veličinami. Nachádza súvislosti medzi závislou premennou
a potenciálnymi vysvetľujúcimi premennými. Používame pri tom vysvetľujúce
premenné, ktoré môžu byť namerané súčasne so závislými premennými alebo aj
premenné z úplne iných zdrojov. Regresná analýza môže byť tiež použitá
na zlúčenie trendu a sezónných zložiek do modelu. Keď je raz model vytvorený,
môže byť použitý na zásah do spomínaných vzťahov alebo, v prípade  dostupnosti
vysvetľujúcich premenných, na vytvorenie predikcie \cite{Liu1992}.

\paragraph{Viacnásobná lineárna regresia}
Viacnásobná regresia sa pokúša modelovať vzťah medzi dvoma alebo viacerými
vysvetľujúcimi premennými a závislou premennou vhodnou lineárnou rovnicou pre
pozorované dáta. Výsledný model je vyjadrený ako funkcia viacerých
vysvetľujúcich premenných. Predpoveďou nie je priamka ako je to pri lineárnej
regresii ale krivku \cite{Grmanova2016}.
Vysvetľujúce premenné môžu predstavovať meteorologické vplyvy, ekonomický
nárast, ceny elektriky či kruzy mien \cite{KumarSingh2013}.

\paragraph{Logistický regresný model}
Nelineárna diskriminantná štatistická metóda. V \textbf{binary response} modely
os y zvyčajne reprezentuje individuálnu alebo experimentálnu jednotku. Y môže
nadobúdať hodnoty 0 alebo 1 pre situácie kedy udalosť nastane alebo nenastane.
Os x reprezentuje vysvetľujúcu veličinu ako vektor, ktorý môže znázorňovať
pravdepodobnosť udalosti (Y = 1) \cite{Li2010}.

%-------------------------------------------------------------------------------
%   Stochastic models
%-------------------------------------------------------------------------------

\subsubsection{Stochastické modely}
Tieto metódy časových radov sú založené na predpoklade, že dáta majú vnútornú
štruktúru, ako napr. autokoreláciu, trend či sezónnu variáciu. Najprv sa
precízne zostaví vzor zodpovedajúci dostupným dátam a potom sa na jeho základe
predpovie budúca hodnota veličiny \cite{KumarSingh2013}.

\paragraph{Autoregressive Moving-Average model}
Model reprezentuje súčasnú hodnotu časového rádu linárne na základe jeho hodnôt
a hodnôt bieleho šumu v predchádzajúcich periódach \cite{KumarSingh2013}.

%-------------------------------------------------------------------------------
%   Support vector regression
%-------------------------------------------------------------------------------

\subsubsection{Support vector regression}
Support Vector Machine a Support Vector Regression sú založené na štatistickej
teórií učenia, nazývanej aj VC teória, podľa svojich autorov, Vapnik
a Chervonenkisa.

Support Vector Machine je použité na množstvo úloh strojového učenia ako je
rozoznávanie vzorov, klasifikácia objektov a v prípade predikcií časových
radov to je regresná analýza. Support Vector Regression je postup, ktorého
funkcia je predpovedaná pomocou nameraných dát, ktorými je Support Vector
Machine postupne natrénované. Toto je odklon od tradičných predpovedí časových
radov, v zmysle že Support Vector Machine nepoužíva žiadny model, ale
predikciu riadia samotné dáta \cite{Sapankevych2009}.

%-------------------------------------------------------------------------------
%   Decision trees
%-------------------------------------------------------------------------------

\subsubsection{Rozhodovacie stromy}
Rozhodovacie stromy sú jednou z najrozšírenejších učiacich metód. Používajú sa
najmä na klasifikáciu. Rozhodovací strom je reprezentovaný ako množina uzlov
a im prislúchajúcich hrán. Uzly reprezentujú atribúty a výstupné hrany sú vždy
označené konkrétnou hodnotou pre atribút, z ktorého vychádzajú. Rozhodovanie
začína v koreni stromu a končí po dosiahnutí listového uzla. Pre riešenie
jedného problému je možné vytvoriť stromy s rôznym počtom a usporiadaním uzlov.
Najlepším riešením je strom s najmenším počtom rozhodovacích
uzlov \cite{Merz1998}.

\paragraph{Regresný rozhodovací strom}

%-------------------------------------------------------------------------------
%   Random forest
%-------------------------------------------------------------------------------

\subsubsection{Random forrest}

%-------------------------------------------------------------------------------
%   Neural networks
%-------------------------------------------------------------------------------

\subsubsection{Neurónové siete}

%-------------------------------------------------------------------------------
%   Ensemble learning
%-------------------------------------------------------------------------------

\subsubsection{Učenie súborov klasifikátorov}
Používa sa na jednodňovú predikciu. Ak \textit{h} je počet meraní, ktoré sú
denne dostupné, v deň \textit{t} sa vykoná \textit{h} predikcií podľa váženého
priemeru \textit{m} modelmi. Nasledujúci deň sa vypočíta chyba predpovede,
na základe ktorej sa znova prepočítajú váhy a každý model sa
aktualizuje\cite{Grmanova2016}.

Učenie súborov klasifikátorov môžeme rozdeliť na homogénne a heterogénne učenie.

\paragraph{Homogénne učenie súborov klasifikátorov}
Pozostáva z modelov rovnakého typu, ktoré sa učia na rôznych podmnožinách
datasetu.
\paragraph{Heterogénne učenie súborov klasifikátorov}
Aplikuje rôzne typy modelov nad rovnakými dátovými množinami\cite{Grmanova2016}.

%-------------------------------------------------------------------------------
%   Exponential smoothing
%-------------------------------------------------------------------------------

\subsubsection{Exponencionálne hladenie}

%-------------------------------------------------------------------------------
%   Naive methods
%-------------------------------------------------------------------------------

\subsubsection{Naivné metódy}
Predpovede sú vytvárané pomocou posledných hodnôt alebo ich priemerov.

\paragraph{Seasonal naïve method}
Poslednú nameranú hodnotu použijeme ako predpoveď pre nasledujúce obdobie. Ak
sú naše dáta vysoko závisle od ročného obdobia, je lepšie použiť na predpoveď
hodnotu z rovnakého obdobia, napr. z minulého roka \cite{Grmanova2016}.

\paragraph{Naïve average long-term method} \label{naive-average}
Predpokladá, že dáta obsahujú vzory, ktoré nie sú závislé od ročných období.
Kvôli tomu sú časové rady lokálne stabilné s pomaly meniacim sa priemerom.
Hodnotu, ktorú použijeme ako predpoveď je iba priemorom viacerých posledných
hodnôt \cite{Grmanova2016}.

\paragraph{Naïve In median long-term method}
Táto metóda je alternativou k \ref{naive-average}. Keďže priemerom nedokáže
model dostatočne rýchlo reagovať na rapídne výkyvy a abnormality, lepšie
výsledky dosiahneme nahradením priemeru za median posledných \textit{n}
meraní \cite{Grmanova2016}.

%-------------------------------------------------------------------------------
%   Analysis of optimizing algorithms
%-------------------------------------------------------------------------------

\subsection{Analýza optimalizačných algoritmov}

% definicia optimalizacnz algoritmov
% preco optimalizacne algoritmy a nie hocco ine

%-------------------------------------------------------------------------------
%   Measurement of prediction accuracy
%-------------------------------------------------------------------------------

\subsection{Meranie presnosti predpovedi}

%-------------------------------------------------------------------------------
%   Chapter 2 - Description of solution
%-------------------------------------------------------------------------------

\newpage
\section{Opis riešenia}

%-------------------------------------------------------------------------------
%   Chapter 3 - Evaluation
%-------------------------------------------------------------------------------

\newpage
\section{Zhodnotenie}

%-------------------------------------------------------------------------------
%   Chapter 4 - Technical documentation
%-------------------------------------------------------------------------------

\newpage
\section{Technická dokumentácia}

% \chapter{Špecifikácia}
% \subsection{Meranie chyby predikcie}



%-------------------------------------------------------------------------------
%   Bibliography
%-------------------------------------------------------------------------------

\newpage
\bibliography{bibliography}
\bibliographystyle{ieeetr}

\end{document}

% \paragraph{Autoregressive model}
% môže modelovať profil záťaže za predpokladu, že zátaž je lineárnou kombináciou
% predchádzajúcich záťaží\cite{KumarSingh2013}.

% \paragraph{Support Vector Machine based Techniques}
% je metóda analyzujúca dáta a rozpoznávajúca vzory, používaná na roztriedenie
% a regresnú analýzu, kombinuje zovšeobecnené riadenie
% s technikou ??????\cite{KumarSingh2013}.
%
% \paragraph{Support Vector Machine}
% je ML algoritmus používaný ako na klasifikáciu tak na regresiu
% support vector sú koordináty jednotlivých meraní napr. muž a žena a ich merané veličny reprezentované na osy, ktoré sú hraničnými elementami rôznych skupín
% maximalizuje rozmädzie medzi support vektormi jednej kategórie a support vektormi druhej kategórie, rozhodovacia funkcia je definovaná podmnožinou testovacej vzorky (jednotlivé supprot vektory)
% v 2D sú kategórie oddelené čiarou vo viacrozmerných dimenziách rovinou
%
% \paragraph{Incremental SVM}
% základom je pridávanie % http://www.jmlr.org/papers/volume7/laskov06a/laskov06a.pdf
% nový bod má najskôr pridelenú váhu 0, ak toto pridelenie nie je optimálnym riešením, teda bod sa môže stať support vectorom,
% váhy ostatných vektorov a rozhodovací prah musia byť aktualizované kvôli získaniu optimálneho riešenia nad novou množinou support vektorov
%
% \paragraph{Linear SVM}
% linárna kombinácia elementov (features, črty) značí, že sa jedná aj o lineárny klasifikátor  % http://stackoverflow.com/questions/6160495/support-vector-machines-a-simple-explanation
% napr ak (w1 * x1 + w2 * x2) > C potom element patrí do skupiny A, hodnotami x1 a x2 je element definovaný, tak ako je bod definovaný x a y súradnicou
% w je váha a C rozhodovacií prah, čiže ak nejaký ohodnotený element neprekročí hranicu spadá do jednej skupiny, ak prekročí spadá do druhej
%
% \paragraph{Concept drift}
% je správanie premennej, ktorú sa snažím predikovať sa môže časom meniť,
% čím sa postupne stáva model menej a menej presný\cite{Grmanova2016}.
%
% \paragraph{Online algorithm}
% spracováva vstup sériovo kúsok po kúsku, vstupné dáta nie sú dostupné na začiatku výpočtu % http://stackoverflow.com/questions/11496013/what-is-the-difference-between-an-on-line-and-off-line-algorithm
% musí spracovať vstup v jednej iterácií bez žiadnej podrobnej znalosti budúcich vstupov % https://xlinux.nist.gov/dads/HTML/online.html
% viac dát, časové obmedzenia, môže sa časom meniť % http://stats.stackexchange.com/questions/897/online-vs-offline-learning
%
% \paragraph{Offline algorithm}
% rieši problém od začiatku so všetkými vstupnými dátami % http://stackoverflow.com/questions/11496013/what-is-the-difference-between-an-on-line-and-off-line-algorithm
% vopred je daná celá séria vstupov % https://xlinux.nist.gov/dads/HTML/offline.html

% \paragraph{Kernel trick}
% problém nie je lineárne separovateľný, originálny nelineárny priestor % http://stats.stackexchange.com/questions/3947/help-me-understand-support-vector-machines
% je premietnutý do viacrozmerného priesotru pomocou nejakej nelineárnej transofrmácia s očakávaním, že to problém už bude riešiteľný
%
% \paragraph{Extreme learning machine}
% je novovznikajúca technika učenia poskytujúca efektívne % http://cherup.yonsei.ac.kr/files/Paper/2013_IEEE%20Intelligent%20Systems%20-%20Off%20line%20version_A%20System%20for%20Signature%20Verification%20Based%20on%20Horizontal%20and%20Vertical%20Components%20in%20Hand%20Gestures.pdf
% a zjednotené riešenie na všeobecné dopredné siete ako
% neurónové siete, RBF siete alebo kernelové učenie

% Časový rád je súbor meraní presne definovaných veličín získavaných opakovanými
% meraniami. Dáta zbierané zriedkavo alebo jednorázovo nepovažujeme za časový rád.
% Pozorované časové rády možno rozložiť na 3 zložky a to trendovú, sezónnu
% a nepravidelnú\cite{AustralianBureau}.

% Trendová zložka predstavuje smer veličiny v dlhodobom horizonte a máva klesajúci
% alebo stúpajúci charakter. Na druhej strane, sezónna zložka má cyklický
% charakter a dĺžka cyklu sa viaže napr. ku dňu, týždnu či roku. Nepravidelná
% zložka reprezentuje náhodné zmeny v prostredí, ktoré nie sú relevantné pre
% predpoveď časových rádov. Pri trénovaní modelu sa ich snažíme odfiltrovať
% optimálnou mierou natrénovania modelu.

% Pri predpovedaní časových radov ako napr. meraní odberu elektriky vznikajú 2 typy tvz. Concept drift.
% \textbf{Concept drift} je zmena správania veličiny, ktorú sa snažíme
% predpovedať. Model sa tak stáva postupne nepresný a je potrebné aby sa tejto
% zmene prispôsobil. Prvým typom je trvalá alebo dočasná zmena spôsobená
% ekonomickými alebo ekologickými faktormi. Druhým typom je sezónna zmena,
% spôsobená zmenami ročných období a množstvom denného svetla. Sezónnu zmenu
% môžeme pozorovať na úrovni dní, týždňov alebo rokov. Kvôli tomu je nutné
% v každom modely rozdeľovať tieto 2 typy concept drift\cite{Grmanova2016}.

% \subsection{Reziduálna zložka}
% Ostáva v časovom rade po odstránení trendovej, cyklickej a sezónnej zložky.
% Je tvorená náhodnými pohybmi v priebehu časového radu. Tiež pokrýva chyby
% v meraní. Obvykle sa predpokladá, že reziduálna zložka je biely šum, teda
% nekorelované náhodné veličny s nulovou strednou hodnotou\cite{http://www.math.sk/mpm/otazka_30.pdf}.

% ε-insensitive loss function defined
% \[
%     L_{\varepsilon}(y, f(x, w)) =
%     \begin{cases}
%       0 \text{ ak } |y - f(x, w)| \leq \varepsilon \\
%       |y - f(x, w)| - \varepsilon \text{ inak } \\
%     \end{cases}
% \]

%-------------------------------------------------------------------------------
%   Chapter X - Conclusion
%-------------------------------------------------------------------------------

% \chapter{Záver}
% Tu bude záver

% Kapitola \chapter{Nazov}
% Necislovana kapitola \chapter*{Nazov}% underline \underline{science}
% Pokapitola (Section) \section{Nazov}
% Subsection \subsection{Nazov}
% Paragraph \paragraph{Nazov}
% Ak niečo nechceme číslovať, použijeme *, avšak, ak to chceme v obsahu, musíme to do neho pridať

% \Huge, \huge, \LARGE, \Large, \large, \normalsize, \small, \footnotesize, \tiny
% italic \textit{accident}.
% bold \textbf{greatest}
% -1 part     1 section     3 subsubsection  5 subparagraph
%  0 chapter  2 subsection  4 paragraph
