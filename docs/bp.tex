% Bakalársky projekt 2016/2017
% Matúš Cuper

%-------------------------------------------------------------------------------
%   PACKAGES AND DOCUMENT CONFIGURATION
%-------------------------------------------------------------------------------

\documentclass[a4paper,slovak,12pt,appendix]{article}

% \usepackage{float}
\usepackage[slovak]{babel}                                                      % title in Slovak
\usepackage[utf8]{inputenc}                                                     % supported special Slovak characters
\usepackage[T1]{fontenc}                                                        % supported word wrapping on the end of line on speciacl Slovak characters
% \usepackage{graphicx}
% \usepackage{url}
\usepackage{times}																															% Times New Roman
\usepackage[unicode]{hyperref}																									% enable hyper references in table of content
\usepackage{indentfirst}                                                        % indent first line after section
\usepackage{amsmath}                                                            % for fractions
\usepackage{amssymb}                                                            % for real numbers sign
\usepackage{algorithm,algpseudocode}                                            % for pseudocode writting
\hypersetup{																																		% with default colours for links
    colorlinks,
		pageanchor=false,
    citecolor=black,
    filecolor=black,
    linkcolor=black,
    urlcolor=black
}

% \usepackage{cite}
% \usepackage{times}
% \usepackage[dvips,dvipdfm,a4paper,centering,textwidth=14cm,top=4.6cm,headsep=.6cm,footnotesep=1cm,footskip=0.6cm,bottom=3.8cm]{geometry}
\usepackage[a4paper, centering,
											left=30mm, top=20mm, right=20mm, bottom=20mm]{geometry}		% set page margins

%-------------------------------------------------------------------------------
%   TITLE PAGES
%-------------------------------------------------------------------------------

\begin{document}
\begin{titlepage}
	\centering
	{\large \textbf{SLOVENSKÁ TECHNICKÁ UNIVERZITA V BRATISLAVE} \par}
	\vspace{0.5cm}
	{\large \textbf{Fakulta informatiky a informačných technológií} \par}
	\vspace{6cm}
	{\huge\bfseries Optimalizácia konfiguračných parametrov predikčných metód \par}
	\vspace{2cm}
	{\scshape\large \textbf{Bakalárska práca} \par}
	\vspace{12.5cm}
	\noindent{\bfseries\large 2016 \hfill Matúš Cuper }
	\vfill
\end{titlepage}

\begin{titlepage}
	\centering
	{\large \textbf{SLOVENSKÁ TECHNICKÁ UNIVERZITA V BRATISLAVE} \par}
	\vspace{0.5cm}
	{\large \textbf{Fakulta informatiky a informačných technológií} \par}
	\vspace{6cm}
	{\huge\bfseries Optimalizácia konfiguračných parametrov predikčných metód \par}
	\vspace{2cm}
	{\scshape\large \textbf{Bakalárska práca} \par}
	\vspace{5.5cm}
	\raggedright
	{\normalsize Študijný program: Informatika \par}
	{\normalsize Číslo študijného odboru: 9.2.1 \par}
	{\normalsize Názov študijného odboru: Informatika \par}
	{\normalsize Školiace pracovisko: Ústav informatiky a softvérového inžinierstva, FIIT STU Bratislava \par}
	{\normalsize Vedúci záverečnej práce: Ing. Marek Lóderer \par}
	\vspace{4.5cm}
	\noindent{\bfseries\large Bratislava 2016 \hfill Matúš Cuper }
\end{titlepage}

%-------------------------------------------------------------------------------
%   ANOTATION
%-------------------------------------------------------------------------------

\newpage
\thispagestyle{plain}
\vspace*{1.5cm}
\begin{center}
  \begin{Large}
    \textbf{Anotácia} \par
  \end{Large}
\end{center}
{Slovenská technická univerzita v Bratislave \par}
{FAKULTA INFORMATIKY A INFORMAČNÝCH TECHNOLÓGIÍ \par}
{Študijný program: Informatika \par}
{Autor: Matúš Cuper \par}
{Bakalárska práca: Optimalizácia konfiguračných parametrov predikčných metód \par}
{Vedúci práce: Ing. Marek Lóderer \par}
{máj 2016 \\} \\
Tu bude text slovenskej anotácie

\newpage
\thispagestyle{plain}
\vspace*{1.5cm}
\begin{center}
  \begin{Large}
    \textbf{Annotation} \par
  \end{Large}
\end{center}
{Slovak University of Technology Bratislava \par}
{FACULTY OF INFORMATICS AND INFORMATION TECHNOLOGIES \par}
{Degree Course: Computer Science \par}
{Author: Matúš Cuper \par}
{Bachelor thesis: Optimizing configuration parameters of prediction methods \par}
{Supervisor: Ing. Marek Lóderer \par}
{May 2016 \\} \\
Tu bude text anglickej anotácie

%-------------------------------------------------------------------------------
%   Declaration
%-------------------------------------------------------------------------------

\newpage
\thispagestyle{plain}
\vspace*{15cm}
\begin{large}
  \noindent \textbf{POĎAKOVANIE} \\
\end{large}
\noindent
Tu bude poďakovanie

\newpage
\thispagestyle{plain}
\vspace*{15cm}
\begin{large}
  \noindent \textbf{ČESTNÉ PREHLÁSENIE} \\
\end{large}
\noindent
Tu bude prehlásenie \\
\vspace*{0.5cm}\\
\hspace*{10cm}............................\\
\hspace*{10.7cm} Matúš Cuper

%-------------------------------------------------------------------------------
%   Table of contents
%-------------------------------------------------------------------------------

\newpage
\tableofcontents

%-------------------------------------------------------------------------------
%   Chapter 1 - Problem analysis
%-------------------------------------------------------------------------------

\newpage
\section{Analýza problému}

%-------------------------------------------------------------------------------
%   Time series
%-------------------------------------------------------------------------------

\subsection{Časové rady}
Časový rad je množina dátových bodov nameraná v čase postupne za sebou.
Matematicky je definovaný ako množina vektorov $x(t)$, kde $t$ reprezentuje
uplynulý čas. Premenná $x(t)$ je považovaná za náhodnú premennú.
Merania v časových radoch sú usporiadané v chronologicky
poradí~\cite{Agrawal2013}.

Časové rady delíme na spojité a diskrétne. Pozorovania pri spojitých časových
radoch sú merané v každej jednotke času, zatiaľ čo diskrétne obsahujú iba
pozorovania v diskrétnych časových bodoch. Hodnoty toku rieky, teploty
či koncentrácie látok pri chemickom procese môžu byť zaznamenané ako spojitý
časový rad. Naopak, populácia mesta, produkcia spoločnosti alebo kurzy mien
reprezentujú diskrétny časový rad. Vtedy sú pozorovania oddelené rovnakými
časovými intervalmi, napr. rokom, mesiacom či dňom~\cite{Agrawal2013}. V našom
prípade sú namerané dáta dostupné každú celú štvrťhodinu.

\subsubsection{Analýza časových radov}
V praxi je vhodný model napasovaný do daného časového radu a zodpovedajúce
parametre sú predpovedané na základe známych dát. Pri predopovedaní časových
radov sú dáta z uplynulých meraní zhromažďované a analyzované za účelom
navrhnutia vhodného matematického modelu, ktorý zachytáva proces generovania
dát pre časové rady. Pomocou tohto modelu sú predpovedané hodnoty budúcich
meraní. Takýto prístup je užitočný, keď nemáme veľa poznatkov o vzore
v meraniach idúcich za sebou alebo máme model, ktorý poskytuje nedostatočne
uspokojivé výsledky~\cite{Agrawal2013}.

Cieľom predikcií časových radov je predpovedať hodnotu premennej v budúcnosti
na základe doteraz nameraných dátových vzoriek. Matematicky zapísané ako
\begin{equation}
  \hat{x}(t+\Delta_t) = f(x(t-a), x(t-b), x(t-c), ...)
  \label{eq-series}
\end{equation}
Hodnota $\hat{x}$ je predpovedaná ako hodnota diskrétneho časového radu $x$.
Preto je potrebné nájsť funkciu $f(x)$, podobnú funkciu $\hat{x}$, ktorá
predpovedá hodnotu časového radu v budúcnosti konzistentne
a objektvíne~\cite{Sapankevych2009}.

Časové rady sú najčastejšie vizualizované ako graf, kde pozorovania sú na
osy $y$ a plynúci čas na osy $x$.

\subsubsection{Zložky časových radov}
Pri predpovedaní časových radov ako napr. meraní odberu elektriky vznikajú
2 typy trendov. Prvým typom je trvalá alebo dočasná zmena spôsobená
ekonomickými alebo ekologickými faktormi. Druhým typom je sezónna zmena,
spôsobená zmenami ročných období a množstvom denného svetla. Môžeme ju pozorovať
na úrovni dňov, týždňov alebo rokov. Veličina, ktorú sa snažíme predpovedať
postupne mení svoje správanie a model sa tak stáva nepresným. Kvôli tomu je
nutné v každom modely rozdeľovať tieto typy tendencií, aby sme vedeli model
zmenám prispôsobiť~\cite{Grmanova2016}.

Vo všeobecnosti sú časové rady zložené zo 4 hlavných zložiek, ktoré môžeme
odlíšiť od pozorovaných dát. Jedná sa o trendovú, cyklickú, sezónnu
a reziduálnu zložku~\cite{Agrawal2013}.

\paragraph{Trendová zložka}
V dlhodobom časovom horizonte majú časové rady tendenciu klesať, rásť alebo
stagnovať. Príkladom môže byť nárast populácie či klesajúca
úmrtnosť~\cite{Agrawal2013}.

\paragraph{Cyklická zložka}
V strednodobom časovom horizonte sa vyskytujú okolnosti, ktoré spôsobujú
cyklické zmeny v časových radoch. Dĺžka periódy je 2 a viac rokov. Táto zložka
je zastúpená najmä pri ekonomických časových radoch napríklad podnikateľský
cyklus pozostávajúci zo 4 fáz, ktoré sa stále opakujú~\cite{Agrawal2013}.

\paragraph{Sezónna zložka}
Ide o kolísanie počas ročných období. Dôležitými faktormi pri tom sú napr.
klimatické podmienky, tradiície alebo počasie. Napríklad predaj zmrzliny sa
v lete zvyšuje, ale počet predaných lyžiarskych súprav klesá~\cite{Agrawal2013}.

\paragraph{Reziduálna zložka}
Jedná sa o veličinu, ktorá nemá žiadny opakovateľný vzor a ani dlhodobý trend.
V časových radoch má nepredvídateľný vplyv na pozorovanú veličinu. V štatistike
zatiaľ nie je definovaná metóda na jej meranie. Označuje sa aj ako náhodná
zložka alebo biely šum. Je spôspobená nepredvídateľnými a nepravideľnými
udalosťami~\cite{Agrawal2013}.

Vo všeobecnosti sa pre tieto 4 zložky používajú 2 rôzne modely. Je to
multiplikatívny model a aditívny model.
\begin{equation}
  \begin{split}
    Y(t) = T(t) \times S(t) \times C(t) \times I(t)
    \\
    Y(t) = T(t) + S(t) + C(t) + I(t)
  \end{split}
  \label{eq-ts-models}
\end{equation}
Vo vzorci~\ref{eq-ts-models} predstavuje $Y(t)$ meranie v čase $t$. Premenné
$T(t)$, $S(t)$, $C(t)$ a $I(t)$ sú zložkami trendu, sezónnosti,
cyklu a náhodnosti. Multiplikatívny model je založený na predpoklade, že časové
rady môžu byť na sebe závislé a môžu byť ovplyvňované medzi sebou, zatiaľ čo
aditívny model predpokladá nezávislosť zložiek~\cite{Agrawal2013}.

%-------------------------------------------------------------------------------
%   Analysis of prediction algorithms
%-------------------------------------------------------------------------------

\subsection{Analýza predičkných algoritmov}
Na základe množstva predikčných

%-------------------------------------------------------------------------------
%   Linear regression
%-------------------------------------------------------------------------------

\subsubsection{Lineárna regresia}
Najpoužívanejšia štatistická metóda, ktorá modeluje vzťah závislej premennej
a vysvetľujúcej premmennej. Závislú premmenu predstavuje veličina, ktorú sa
snažíme predpoved, čo je v našom prípade spotreba elektriky. Vysvetľujúca
premenná v sebe zahŕňa rôzne faktory, ktoré ovplyvňujú závislú premennú.
Môžeme si pod tým predstaviť deň v týždni, počasie, tradície alebo rôzne
udalosti, ktoré majú vplyv na predpoveď.~\cite{KumarSingh2013}.

Predpokladajme typický regresný problém. Dáta pozostávajúce z množiny \textit{n}
meraní majú formát $\left\{(x_1, f(x_1)), ..., (x_n, f(x_n))\right\}$.
Úlohou regresie je odvodiť funkciu $\hat{f}$ z dát, kde
\begin{equation}
  \hat{f} : X \to \mathbb{R} \text{, kde, } \hat{f}(x) = f(x), \forall x \in X,
  \label{eq-regresia}
\end{equation}
Funkcia $f$ vo vzorci~\ref{eq-regresia} reprezentuje reálnu neznámu
funkciu. Algoritmus použitý na odvodenie funkcie $\hat{f}$ sa nazýva
indukčný algoritmus alebo žiak. Funkcia $\hat{f}$ sa nazýva model alebo
prediktor. Obvykle je úlohou regresie minimalizovať odchýlku funkcie pre
štvorcovú chybu, konkrétne strednú štvorcovú chybu MSE~\cite{Mendes-Moreira2012}.

Keďže časový rad pozostáva z viacerých zložiek, môžeme ho zapísať ako funkciu
$L(t)$ definovanú ako
\begin{equation}
  L(t) = Ln(t) + \sum a_i x_i(t) + e(t)
  \label{eq-odber}
\end{equation}
Vo vzorci~\ref{eq-odber} funkcia $Ln(t)$ predstavuje odber elektriky v čase
$t$. Hodnota $a_i$ je odhadovaný pomaly meniaci sa koeficient. Faktory
$x_i(t)$ nezávisle vplývajú na spotrebu elektriky. Môže sa jednať napríklad
o počasie alebo zvyky ľudí. Komponent $e(t)$ je biely šum, ktorý má nulovú
strednú hodnotu a pevnú varianciu. Číslo $n$ je počet meraní, obvykle 24
alebo 168, v našom prípade 96 meraní počas jedného dňa~\cite{KumarSingh2013}.

\paragraph{Lineárna regresná analýza}
Regresná analýza je štatistická metóda používaná na modelovanie vzťahov, ktoré
môžu existovať medzi veličinami. Nachádza súvislosti medzi závislou premennou
a potenciálnymi vysvetľujúcimi premennými. Používame pri tom vysvetľujúce
premenné, ktoré môžu byť namerané súčasne so závislými premennými alebo aj
premenné z úplne iných zdrojov. Regresná analýza môže byť tiež použitá
na zlúčenie trendu a sezónných zložiek do modelu. Keď je raz model vytvorený,
môže byť použitý na zásah do spomínaných vzťahov alebo, v prípade  dostupnosti
vysvetľujúcich premenných, na vytvorenie predikcie~\cite{Liu1992}.

\paragraph{Viacnásobná lineárna regresia}
Viacnásobná regresia sa pokúša modelovať vzťah medzi dvoma alebo viacerými
vysvetľujúcimi premennými a závislou premennou vhodnou lineárnou rovnicou pre
pozorované dáta. Výsledný model je vyjadrený ako funkcia viacerých
vysvetľujúcich premenných~\cite{Grmanova2016}.

Túto funkciu môžeme zapísať ako
\begin{equation}
  Y(t) = V_t a_t + e_t
  \label{eq-multi-regresia}
\end{equation}
Vo vzorci~\ref{eq-multi-regresia} $t$ označuje čas, kedy bolo meranie
uskutočnené. $Y(t)$ predstavuje celkový nameraný odber elektriky. Vektor $V_t$
reprezentuje hodnoty vysvetľujúcich premenných v čase merania. Vysvetľujúce
premenné môžu predstavovať meteorologické vplyvy, ekonomický nárast, ceny
elektriky či kruzy mien. Chybu modelu v čase $t$ zapíšeme
ako $e_t$~\cite{KumarSingh2013}.

\paragraph{Logistický regresný model}
Nelineárna diskriminantná štatistická metóda. V \textbf{binary response} modely
os $y$ zvyčajne reprezentuje individuálnu alebo experimentálnu jednotku. $Y$ môže
nadobúdať hodnoty 0 alebo 1 pre situácie kedy udalosť nastane alebo nenastane.
Os $x$ reprezentuje vysvetľujúcu veličinu ako vektor, ktorý môže znázorňovať
pravdepodobnosť udalosti $(Y = 1)$~\cite{Li2010}.

%-------------------------------------------------------------------------------
%   Stochastic models
%-------------------------------------------------------------------------------

\subsubsection{Stochastické modely}
Tieto metódy časových radov sú založené na predpoklade, že dáta majú vnútornú
štruktúru, ako napr. autokoreláciu, trend či sezónnu variáciu. Najprv sa
precízne zostaví vzor zodpovedajúci dostupným dátam a potom sa na jeho základe
predpovie budúca hodnota veličiny~\cite{KumarSingh2013}.

\paragraph{Autoregresný model}
V autoregresívnom modely je budúca hodnota premennej predpokladaná ako súčet
lineárnej kombinácie $p$ predchdzajúcich meraní, náhodnej chyby a konštanty.
Matematicky môžeme autoregresný model zapisať ako
\begin{equation}
  y_t = c + \sum_{i=1}^{p} \varphi_i y_{t-i} + \varepsilon_t
  \label{eq-ar}
\end{equation}
Vo vzorci~\ref{eq-ar} hodnota $y_t$ predstavuje predpovedanú hodnotu
v čase $t$. Náhodnú chybu v čase $t$ zapíšeme ako $\varepsilon_t$. Hodnoty
$\varphi_i$ sú parametre modelu a $c$ je konštanta. Konštantou $p$ označujeme
rad modelu~\cite{Agrawal2013}.

\paragraph{Model kĺzavého priemeru}
Model kĺzavého priemeru na rozdiel od autoregresného modelu používa ako
vysvetľujúce premenné chyby predchádzajúcich meraní a nie priamo hodnoty.
Matematicky môžeme tento vzťah zapísať ako
\begin{equation}
  y_t = \mu + \sum_{j=1}^{q} \Theta_j \varepsilon_{t-j} + \varepsilon_t
  \label{eq-ma}
\end{equation}
Vo vzorci~\ref{eq-ma} hodnota $y_t$ predstavuje strednú hodnotu
postupnosti meraní v čase $t$. Hodnoty $\Theta_j$ sú parametre modelu
a konštantou $q$ označujeme rad modelu. Vychádzame z predpokladu, že náhodná
zložka $\varepsilon_t$ je biely šum, čo je rovnomerne distribuovaná náhodná
premenná, ktorá má nulovú strednú hodnotu a konštantnú varianciu
$\sigma^2$~\cite{Agrawal2013}.

\paragraph{Autoregressive Moving-Average model}
Model reprezentuje súčasnú hodnotu časového rádu linárne na základe jeho hodnôt
a hodnôt bieleho šumu v predchádzajúcich periódach~\cite{KumarSingh2013}.

Ide o kombináciu autoregresie (AR) a kĺzavého priemeru (MA), vhodnú pre
modelovanie jednorozmerných časových radov. Matematicky môžeme reprezentovať
tento model ako súčet predchádzajúcich modelov
\begin{equation}
  y_t = c + \varepsilon_t + \sum_{i=1}^{p} \varphi_i y_{t-i}  \sum_{j=1}^{q} \Theta_j \varepsilon_{t-j}
  \label{eq-arma}
\end{equation}
Rad modelu určuje $p$ a $q$~\cite{Agrawal2013}.

\paragraph{Autoregressive Integrated Moving-Average model}
Modely typu ARMA môžu byť použité iba na statické časové rady. Mnoho časových
rádov v praxi vykazuje nestatické správanie a tiež tie, ktoré obsahujú
komponenty trendu a sezónnosti. Kvôli tomu bol navrhnutý model ARIMA, ktorý je
generalizáciou modelu ARMA a zahŕňa tak v sebe aj prípady nestatických časových
rádov. Z nestatických časových radov sa vytvárajú statické pomocou konečného
počtu derivovaní dátových bodov. Vzniká tak matematický model, ktorý môžeme
zapísať ako
\begin{equation}
  \Big( 1 - \sum_{i=1}^{p} \varphi_i L^i \Big) (1-L)^d y_t = \Big( 1 + \sum_{j=1}^{q} \Theta_j L^j \Big) + \varepsilon_t
  \label{eq-arima}
\end{equation}
Vzorec~\ref{eq-arima} môžeme zapísať aj jednoduchšie a to
\begin{equation}
  \varphi(L) (1-L)^d y_t = \Theta(L) \varepsilon_t
  \label{eq-arima-short}
\end{equation}
Vo vzorci~\ref{eq-arima} predstavujú premenné $p$, $d$ a $q$ rad autoregresného
modelu, modelu kĺzavého priemeru a integrovaného modelu. Hodnota $d$ zodpovedá
stupňu derivovania, zvyčajne je rovná 1. V prípade, že $d=0$ dostaneme klasický
ARMA model. Rovnakým spôsobom vieme dostať modely AR a MA~\cite{Agrawal2013}.

%-------------------------------------------------------------------------------
%   Support vector regression
%-------------------------------------------------------------------------------

\subsubsection{Support vector regression}
Support Vector Machine a Support Vector Regression sú založené na štatistickej
teórií učenia, nazývanej aj VC teória, podľa svojich autorov, Vapnik
a Chervonenkisa.

Support Vector Machine je použité na množstvo úloh strojového učenia ako je
rozoznávanie vzorov, klasifikácia objektov a v prípade predikcií časových
radov to je regresná analýza. Support Vector Regression je postup, ktorého
funkcia je predpovedaná pomocou nameraných dát, ktorými je Support Vector
Machine postupne natrénované. Toto je odklon od tradičných predpovedí časových
radov, v zmysle že Support Vector Machine nepoužíva žiadny model, ale
predikciu riadia samotné dáta~\cite{Sapankevych2009}.

Táto predikčná metóda nie ja závislá na modely ani na žiadnych lineárnych
procesoch. Tiež poskytuje malý počet voľných parametrov. Garantuje konvergenciu
k ideálnemu riešeniu a môže byť výpočtovo efektívna~\cite{Sapankevych2009}.

%-------------------------------------------------------------------------------
%   Decision trees
%-------------------------------------------------------------------------------

\subsubsection{Rozhodovacie stromy}
Rozhodovacie stromy sú jednou z najrozšírenejších učiacich metód. Používajú sa
najmä na klasifikáciu. Rozhodovací strom je reprezentovaný ako množina uzlov
a im prislúchajúcich hrán. Uzly reprezentujú atribúty a výstupné hrany sú vždy
označené konkrétnou hodnotou pre atribút, z ktorého vychádzajú. Rozhodovanie
začína v koreni stromu a končí po dosiahnutí listového uzla. Pre riešenie
jedného problému je možné vytvoriť stromy s rôznym počtom a usporiadaním uzlov.
Najlepším riešením je strom s najmenším počtom rozhodovacích
uzlov~\cite{Merz1998}.

\paragraph{Regresný rozhodovací strom}

%-------------------------------------------------------------------------------
%   Random forest
%-------------------------------------------------------------------------------

\subsubsection{Random forrest}

%-------------------------------------------------------------------------------
%   Neural networks
%-------------------------------------------------------------------------------

\subsubsection{Neurónové siete}
Je veľa typov neurónových sietí napríklad viacvrstvové perceptónové siete,
samoriadiace siete, siete s viacerými skrytými vrstvami atď. V každej skrytej
vrstve je množstvo neurónov. Hlavnou výhodou je, že väčšina sietí nepotrebuje
model. Na druhej strane, trénovanie obvykle zaberá veľa času. Výstupom siete
je lineárna rovnica váh prepojených so vstupom~\cite{KumarSingh2013}.

Najpoužívanejšou neurónovou sietou je viacvrstvový perceptón, ktorý pozostáva
z uzlov a im prislúchajúcim hranám. Uzly sú zoskupované do rôznych vrstiev.
Prvá vrstva je vstupná vrstva, kde počet $d$ označuje počet vstupných parametrov
vstupujúcich do siete. Táto vrstva je následne prepojená hranami so skrytou
vrstvou pozostávajúcou z $h$ uzlov. Tá je potom prepojená s výstupnou vrstvou
s $c$ uzlami. Kvôli tomu sa tieto siete zvyknú označovať aj ako dopredné
siete~\cite{Merz1998}.

Elementy skrytých a výstupných vrstiev sú umelé neuróny pozostávajúce z uzlov,
viacerých vstupujúcich a jednou výstpnou hranou. Funkciou neurónu je
transformovať lineárnu kombináciu vstupov pomocou nelineárnej aktivačnej
funkcie, čiže každú vstupnú hranu prenásobiť jej váhou a výsledok týchto súčinou
sčítať. Tak dostaneme pre neurón $j$ vzorec~\ref{eq-linear-combination}
opisujúci lineárnu kombináciu vstupov $a_j$
\begin{equation}
  a_j = \sum_{i=1}^{d} w_{ji} x_i
  \label{eq-linear-combination}
\end{equation}
Príčom váha $w_{ji}$ označuje váhu medzi neurónom $i$ na vstupnej vrstve
a neurónom $j$ na skrytej vrstve~\cite{Merz1998}.

Aktivovanie neurónu $j$ závisí od jeho aktivačnej funkcie $g(a_j)$. Jednu
z najpoužívanejších aktivačných funkcií, logistickú sigmoidnú funkciu, môžeme
matematicky zapísať ako
\begin{equation}
  g(a) \equiv \frac{1}{1 + exp(-a)}
  \label{eq-sigmoid-function}
\end{equation}
Je zrejmé, že funkcia zo vzoraca~\ref{eq-sigmoid-function} vracia hodnoty
v rozmedzí $(0,1)$~\cite{Merz1998}.

%-------------------------------------------------------------------------------
%   Ensemble learning
%-------------------------------------------------------------------------------

\subsubsection{Učenie súborov klasifikátorov}
Používa sa na jednodňovú predikciu. Ak \textit{h} je počet meraní, ktoré sú
denne dostupné, v deň \textit{t} sa vykoná \textit{h} predikcií podľa váženého
priemeru \textit{m} modelmi. Nasledujúci deň sa vypočíta chyba predpovede,
na základe ktorej sa znova prepočítajú váhy a každý model sa
aktualizuje\cite{Grmanova2016}.

Učenie súborov klasifikátorov môžeme rozdeliť na homogénne a heterogénne učenie.

\paragraph{Homogénne učenie súborov klasifikátorov}
Pozostáva z modelov rovnakého typu, ktoré sa učia na rôznych podmnožinách
datasetu.
\paragraph{Heterogénne učenie súborov klasifikátorov}
Aplikuje rôzne typy modelov nad rovnakými dátovými množinami\cite{Grmanova2016}.

%-------------------------------------------------------------------------------
%   Exponential smoothing
%-------------------------------------------------------------------------------

\subsubsection{Exponencionálne hladenie}

%-------------------------------------------------------------------------------
%   Naive methods
%-------------------------------------------------------------------------------

\subsubsection{Naivné metódy}
Predpovede sú vytvárané pomocou posledných hodnôt alebo ich priemerov.

\paragraph{Seasonal naïve method}
Poslednú nameranú hodnotu použijeme ako predpoveď pre nasledujúce obdobie. Ak
sú naše dáta vysoko závisle od ročného obdobia, je lepšie použiť na predpoveď
hodnotu z rovnakého obdobia, napr. z minulého roka~\cite{Grmanova2016}.

\paragraph{Naïve average long-term method}
Predpokladá, že dáta obsahujú vzory, ktoré nie sú závislé od ročných období.
Kvôli tomu sú časové rady lokálne stabilné s pomaly meniacim sa priemerom.
Hodnotu, ktorú použijeme ako predpoveď je iba priemorom viacerých posledných
hodnôt~\cite{Grmanova2016}.

\paragraph{Naïve In median long-term method}
Táto metóda je alternativou k predchádzajúcej metóde. Keďže priemerom nedokáže
model dostatočne rýchlo reagovať na rapídne výkyvy a abnormality, lepšie
výsledky dosiahneme nahradením priemeru za median posledných \textit{n}
meraní~\cite{Grmanova2016}.

%-------------------------------------------------------------------------------
%   Analysis of optimizing algorithms
%-------------------------------------------------------------------------------

\subsection{Analýza optimalizačných algoritmov}
Genetické algoritmy sú bilogicky inšpirované algoritmy patriace do triedy
evolučných algoritmov. Genetické algoritmy sú stochastické optimalizačné
algortimy s \textbf{global search potential}. Od tradičných algoritmov sa líšia
hlavne v počte riešení, ktoré sú kandidátmi na najlepšie riešenie. Tradičné
vyhľadávacie algoritmy prehľadávajú dôkladne iba jedno riešenie, zatiaľ čo
genetické algoritmy hlbšie spracujú viacero kandidátou naraz. Každý kandidát na
optimálne riešenie problému je reprezentovaný dátovou štruktúrou, ktorú
označujeme pojmom jedinec. Súbor jedincov tvorí populáciu. Začiatok procesu
začína náhodnými riešeniami populácie, ktorý sa postupne
vylepšuje~\cite{Chavan2015}.

Pri genetických algoritmoch sa zavádzajú pojmy ako chromozón, fitness funkcia,
kríženie, operátor reprodukcie či mutácie~\cite{Chavan2015}.

\paragraph{Chromozón}
\paragraph{Fitness funkcia}
je funkcia určujúca efektívnosť chromozónu, pre ktoré je vypočítaná fitness
funkcia. Rovnica \ref{eq-fitness} predstavuej porovnanie fitness funkcie
aktuálneho chromozónu a fitness funkcie cieľového chromozónu
\begin{equation}
  fitness = \delta + P
  \label{eq-fitness}
\end{equation}
Faktor $\delta$ predstavuje
~\cite{Chavan2015}.

\paragraph{Operátor reprodukcie}
je obvykle prvý operátor, ktorý sa uplatní na populáciu. Operátor náhodne
vyberie reťazce z dvoch chromozónov na párenie~\cite{Chavan2015}.

\paragraph{Kríženie}
je operátor \textbf{recombination}. Kríženie vykonáva výmenu blokov chromozónov.
Z druhého chromozónu je vybraný reťazec náhodnej veľkosti, ktorý sa vymení
s rovnako dlhým reťazcom z prvého chromozónu~\cite{Chavan2015}.

\paragraph{Operátor mutácie}
sa vykoná po vykonaní operátora reprodukcie. Mutácia chromozónu invertuje jeden
bit s nízkou pravdepodobnosťou~\cite{Chavan2015}.

Princíp fungovania genetických algoritmov možno znázorniť v nasledujúcom
pseudokóde~\cite{Chavan2015}
\begin{algorithm}
  \caption{Pseudokód genetického algoritmu}
  \begin{algorithmic}[1]
    \State Náhodne inicializovanie jedincov Inicializácia
    \State Vyhodnotenie fitness funkcie pre každého jedinca \label{ga-fitness}
    \State Výber jedincov pre ďalšiu populáciu na základe fitnesss funkcie \label{ga-selection}
    \State Kríženie jedincov \label{ga-crossing}
    \State Mutovanie jedincov \label{ga-mutation}
    \State Kontrola či nebolo nájdené žiadané optimálne riešenie \label{ga-condition}
    \State Ukončenie ak sa našlo takéto riešenie, inak opakovanie krokov \ref{ga-fitness} až \ref{ga-condition}
    \State Koniec
  \end{algorithmic}
\end{algorithm}

\subsubsection{Artifical bee colony}
ABC algoritmus je pomerne nový medzi rojovými algoritmami. Princíp je založený
na biologickom procese, správaní medonosných včiel pri hľadaní potravy. Hlavný
mechanizmus ktorým včely optimalizujú množstvo procesov je \textbf{waggle dance},
ktorým včely lokalizujú zdroje potravy a nachádzajú ďalšie~\cite{Chavan2015}.

Každá včela na pracujúca v roji sa spolupodiela na tvorbe celého systému na
globálnej úrovni. Správanie systému je určené lokálnym správaním, kde spolupráca
a zladenie jedincov vedie k štruktúrovanému kolaboračnému systému~\cite{Chavan2015}.

Algoritmus funguje na princípe, že včely nájdu najviac výnosný zdroj
s použitím, čo najmenšej energie. \textbf{Foragers} (zrejme robotnice hľadajúce zdroj jedla) uvažujú
presúvanie sa medzi zdrojmi nektárov na základe kvality alebo zisku zdroju.
Algoritmus poskytuje samo-manažovateľné a samo-organizované riešenie, vo svojej
podstate decentralizované, pre daný problém~\cite{Buhussain2016}.

\subsubsection{Kolónia mravcov}
Pri tomto algoritme, mravce tiež opúšťajú mravenisko, kvôli hľadaniu zdrojov
potravy náhodne. Potom vyhodnotia kvalitu zdroja potravy a donesú ho naspäť do
mraveniska. Zanechávajú pri tom na zemi chemické stopy. Sila týchto stôp závisí
od kvality nájdeného zdroja potravy. Mnoho výskumov využíva tento algoritmus na
riešenie NP problémov, ako napríklad problém obchodných cestujúcich,
vyfarbovanie grafov, smerovnaie áut alebo plánovacie problémy. Používa sa aj pri
\textbf{cloud computing} na nájdenie optimálneho riešenia pri plánovaní úloh
pre virtuálne servery~\cite{Buhussain2016}.

Keď mravce hľadajú potravu prvý krát, hľadajú náhodne až kým nenájdu zdroj
potravy. Zanechávajú pri tom za sebou chemickú stopu nazývanú feromón, ktorá
tak vedie k zdroju. Tá následne priťahuje ostatné mravce k tomuto zdroju
potravy. Tento proces pokračuje pokiaľ mravce nenájdu najkratšiu cestu vedúcu
ku konkrétnemu zdroju potravy. Najkratšia cesta je určená naakumulovaným
množstvom feromónov na ceste k zdroju potravy~\cite{Buhussain2016}.

% definicia optimalizacnz algoritmov
% preco optimalizacne algoritmy a nie hocco ine

%-------------------------------------------------------------------------------
%   Measurement of prediction accuracy
%-------------------------------------------------------------------------------

\subsection{Meranie presnosti predpovedi}
Pre vyhodnotenie efektívnosti a presnosti modelov je potrebné merať ich
vlastnosti tak, aby sme ich vedeli medzi sebou porovnáva. V nasledujúcich
spôsoboch merania sú použité pojmy ako aktuálna hodnota $y_t$, predpovedaná
hodnota $f_t$ alebo chyba predpovede $e_t$ definovaná ako $e_t = y_t - f_t$.
Veľkosť testovacej množiny budeme označovať ako $n$~\cite{Agrawal2013}.

\subsubsection{Stredná chyba predpovede}
V anglickej literatúre označovaná ako MFE. Matematickú funkciu zapísať ako
\begin{equation}
  MFE = \frac{1}{n} \sum_{t=1}^{n} e_t
  \label{eq-mfe}
\end{equation}
Týmto spôsobom meriame priemernú odchýlku predpovedanej hodnoty od aktuálnej.
Zistíme tak smer chyby nazývaný tiež \textbf{Forecast bias}. Nevýhodou je, že
kladné a záporné chyby sa vynulujú a potom nie je možné zistiť presnú hodnotu
chyby. Pri namerani extrémnych chýb, nie sú nijak špeciálne penalizované.
Taktiež hodnota chyby závisí od škály meraní a môže byť ovplyvnená aj
transformáciami dát. Dobré predpovede majú hodnotu blízku 0~\cite{Agrawal2013}.

\subsubsection{Stredná absolútna chyba}
V anglickej literatúre označovaná ako MAE. Patrí k jedným z najpoužívanejších.
Funkciu môžeme zapísať ako
\begin{equation}
  MAE = \frac{1}{n} \sum_{t=1}^{n} |e_t|
  \label{eq-mae}
\end{equation}
Týmto spôsobom meriame priemernú absolútnu odchýlku predpovedanej hodnoty od
aktuálne. Zistíme tak celkový rozsah chyby, ktorá nastala počas predpovede.
Narozdiel od merania chyby pomocou vzorca~\ref{eq-mfe} sa kladné a záporné
chyby nevynulujú, čo má však za následok, že nevieme určiť celkový smer chyby.
Na druhej strane tiež nenastáva žiadna penalizácia pri extrémnych chybách,
hodnota chyby závisí od škály meraní, môže byť ovplynená transformáciami dát
a dobré predpovede majú hodnotu čo najbližšiu 0~\cite{Agrawal2013, Gutierrez2015}.

\subsubsection{Stredná aboslútna percentuálna chyba}
V anglickej literatúre označovaná ako MAPE. Matematicky môžeme túto funkciu
zapísať ako
\begin{equation}
  MAPE = \frac{1}{n} \sum_{t=1}^{n} \Big|\frac{e_t}{y_t}\Big| \times 100
  \label{eq-mape}
\end{equation}
Pomocou tohto merania chyby získavame percentuálny prehľad o priemernej
absolútnej chybe, ktorá sa vyskytla počas predpovedi. Veľkosť chyby nezávisí od
škály merania, ale je závislá od transformácií dát. Tiež nie je možné zistiť
smer chyby a ani nenastáva žiadna penalizácia pri extrémnych
chybách~\cite{Agrawal2013}.

\subsubsection{Stredná percentuálna chyba}
V anglickej literatúre označovaná ako MPE

\subsubsection{Stredná štvorcová chyba}
V anglickej literatúre označovaná ako MSE

%-------------------------------------------------------------------------------
%   Chapter 2 - Description of solution
%-------------------------------------------------------------------------------

\newpage
\section{Opis riešenia}

%-------------------------------------------------------------------------------
%   Chapter 3 - Evaluation
%-------------------------------------------------------------------------------

\newpage
\section{Zhodnotenie}

%-------------------------------------------------------------------------------
%   Chapter 4 - Technical documentation
%-------------------------------------------------------------------------------

\newpage
\section{Technická dokumentácia}

% \chapter{Špecifikácia}
% \subsection{Meranie chyby predikcie}



%-------------------------------------------------------------------------------
%   Bibliography
%-------------------------------------------------------------------------------

\newpage
\bibliography{bibliography}
\bibliographystyle{ieeetr}

\end{document}

% \paragraph{Autoregressive model}
% môže modelovať profil záťaže za predpokladu, že zátaž je lineárnou kombináciou
% predchádzajúcich záťaží\cite{KumarSingh2013}.

% \paragraph{Support Vector Machine based Techniques}
% je metóda analyzujúca dáta a rozpoznávajúca vzory, používaná na roztriedenie
% a regresnú analýzu, kombinuje zovšeobecnené riadenie
% s technikou ??????\cite{KumarSingh2013}.
%
% \paragraph{Support Vector Machine}
% je ML algoritmus používaný ako na klasifikáciu tak na regresiu
% support vector sú koordináty jednotlivých meraní napr. muž a žena a ich merané veličny reprezentované na osy, ktoré sú hraničnými elementami rôznych skupín
% maximalizuje rozmädzie medzi support vektormi jednej kategórie a support vektormi druhej kategórie, rozhodovacia funkcia je definovaná podmnožinou testovacej vzorky (jednotlivé supprot vektory)
% v 2D sú kategórie oddelené čiarou vo viacrozmerných dimenziách rovinou
%
% \paragraph{Incremental SVM}
% základom je pridávanie % http://www.jmlr.org/papers/volume7/laskov06a/laskov06a.pdf
% nový bod má najskôr pridelenú váhu 0, ak toto pridelenie nie je optimálnym riešením, teda bod sa môže stať support vectorom,
% váhy ostatných vektorov a rozhodovací prah musia byť aktualizované kvôli získaniu optimálneho riešenia nad novou množinou support vektorov
%
% \paragraph{Linear SVM}
% linárna kombinácia elementov (features, črty) značí, že sa jedná aj o lineárny klasifikátor  % http://stackoverflow.com/questions/6160495/support-vector-machines-a-simple-explanation
% napr ak (w1 * x1 + w2 * x2) > C potom element patrí do skupiny A, hodnotami x1 a x2 je element definovaný, tak ako je bod definovaný x a y súradnicou
% w je váha a C rozhodovacií prah, čiže ak nejaký ohodnotený element neprekročí hranicu spadá do jednej skupiny, ak prekročí spadá do druhej
%
% \paragraph{Concept drift}
% je správanie premennej, ktorú sa snažím predikovať sa môže časom meniť,
% čím sa postupne stáva model menej a menej presný\cite{Grmanova2016}.
%
% \paragraph{Online algorithm}
% spracováva vstup sériovo kúsok po kúsku, vstupné dáta nie sú dostupné na začiatku výpočtu % http://stackoverflow.com/questions/11496013/what-is-the-difference-between-an-on-line-and-off-line-algorithm
% musí spracovať vstup v jednej iterácií bez žiadnej podrobnej znalosti budúcich vstupov % https://xlinux.nist.gov/dads/HTML/online.html
% viac dát, časové obmedzenia, môže sa časom meniť % http://stats.stackexchange.com/questions/897/online-vs-offline-learning
%
% \paragraph{Offline algorithm}
% rieši problém od začiatku so všetkými vstupnými dátami % http://stackoverflow.com/questions/11496013/what-is-the-difference-between-an-on-line-and-off-line-algorithm
% vopred je daná celá séria vstupov % https://xlinux.nist.gov/dads/HTML/offline.html

% \paragraph{Kernel trick}
% problém nie je lineárne separovateľný, originálny nelineárny priestor % http://stats.stackexchange.com/questions/3947/help-me-understand-support-vector-machines
% je premietnutý do viacrozmerného priesotru pomocou nejakej nelineárnej transofrmácia s očakávaním, že to problém už bude riešiteľný
%
% \paragraph{Extreme learning machine}
% je novovznikajúca technika učenia poskytujúca efektívne % http://cherup.yonsei.ac.kr/files/Paper/2013_IEEE%20Intelligent%20Systems%20-%20Off%20line%20version_A%20System%20for%20Signature%20Verification%20Based%20on%20Horizontal%20and%20Vertical%20Components%20in%20Hand%20Gestures.pdf
% a zjednotené riešenie na všeobecné dopredné siete ako
% neurónové siete, RBF siete alebo kernelové učenie

% Časový rád je súbor meraní presne definovaných veličín získavaných opakovanými
% meraniami. Dáta zbierané zriedkavo alebo jednorázovo nepovažujeme za časový rád.
% Pozorované časové rády možno rozložiť na 3 zložky a to trendovú, sezónnu
% a nepravidelnú\cite{AustralianBureau}.

% Trendová zložka predstavuje smer veličiny v dlhodobom horizonte a máva klesajúci
% alebo stúpajúci charakter. Na druhej strane, sezónna zložka má cyklický
% charakter a dĺžka cyklu sa viaže napr. ku dňu, týždnu či roku. Nepravidelná
% zložka reprezentuje náhodné zmeny v prostredí, ktoré nie sú relevantné pre
% predpoveď časových rádov. Pri trénovaní modelu sa ich snažíme odfiltrovať
% optimálnou mierou natrénovania modelu.

% Pri predpovedaní časových radov ako napr. meraní odberu elektriky vznikajú 2 typy tvz. Concept drift.
% \textbf{Concept drift} je zmena správania veličiny, ktorú sa snažíme
% predpovedať. Model sa tak stáva postupne nepresný a je potrebné aby sa tejto
% zmene prispôsobil. Prvým typom je trvalá alebo dočasná zmena spôsobená
% ekonomickými alebo ekologickými faktormi. Druhým typom je sezónna zmena,
% spôsobená zmenami ročných období a množstvom denného svetla. Sezónnu zmenu
% môžeme pozorovať na úrovni dní, týždňov alebo rokov. Kvôli tomu je nutné
% v každom modely rozdeľovať tieto 2 typy concept drift\cite{Grmanova2016}.

% \subsection{Reziduálna zložka}
% Ostáva v časovom rade po odstránení trendovej, cyklickej a sezónnej zložky.
% Je tvorená náhodnými pohybmi v priebehu časového radu. Tiež pokrýva chyby
% v meraní. Obvykle sa predpokladá, že reziduálna zložka je biely šum, teda
% nekorelované náhodné veličny s nulovou strednou hodnotou\cite{http://www.math.sk/mpm/otazka_30.pdf}.

% ε-insensitive loss function defined
% \[
%     L_{\varepsilon}(y, f(x, w)) =
%     \begin{cases}
%       0 \text{ ak } |y - f(x, w)| \leq \varepsilon \\
%       |y - f(x, w)| - \varepsilon \text{ inak } \\
%     \end{cases}
% \]

%-------------------------------------------------------------------------------
%   Chapter X - Conclusion
%-------------------------------------------------------------------------------

% \chapter{Záver}
% Tu bude záver

% Kapitola \chapter{Nazov}
% Necislovana kapitola \chapter*{Nazov}% underline \underline{science}
% Pokapitola (Section) \section{Nazov}
% Subsection \subsection{Nazov}
% Paragraph \paragraph{Nazov}
% Ak niečo nechceme číslovať, použijeme *, avšak, ak to chceme v obsahu, musíme to do neho pridať

% \Huge, \huge, \LARGE, \Large, \large, \normalsize, \small, \footnotesize, \tiny
% italic \textit{accident}.
% bold \textbf{greatest}
% -1 part     1 section     3 subsubsection  5 subparagraph
%  0 chapter  2 subsection  4 paragraph
